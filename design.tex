\section{Conceptualisation and design}
At the~heart of user interface design is the~"user" and the~user is in the~"driver's seat". Therefeore it is critical that users of any virtual environment should be able to use the~interface intuitively, regardless of cultural diversities. the~goal of user interface design is to make the~user's interaction as simple and efficient as possible, in terms of accomplishing user goals. This is called user-centered design. \cite{badrul}

The~most important goal for designing virtual environment in MathworldVR is to reach learners more effectively through seamless integration of content and its organization, together with the~navigational and interactive controls that user use to work with the~content.


\subsection{Inputs and objections}
MathworldVR is designed as a~client-side, single page WebVR application. That means user doesn't have to install any additional software besides the~supported experimental browser (this will change in future and user's would not need to install any additional browsers since Mozilla Firefox will introduce full WebVR support in version 55 and Google Chrome should introduce WebVR support in version 59). To be able to use MathworldVR properly, user needs to have on of HTC Vive or Oculus Touch headset installed and prepared. It can be used also on any device through any browser supporting WebVR API in version 1.0 and up. If accessed through the~mobile device, user can look around the~3D environment but is not allowed to move from current position.

Code of the~application will be fully open-sourced under the~MIT license, hosted on \url{GitHub.com}, an~online version control repository. That will allow more skilled developers from around the~world contribute to the~project and provide easier working conditions with various universities.

When user opens MathworldVR website, she should be informed about supported browsers and provided with the~links to check her browser's WebVR support.

\subsection{Functionality planning}
MathworldVR will help students to explore various parametrized functions in 3D environment with use of VR hand controllers interactions. It will contain this functionality:

\begin{itemize}
\item{Visualization of various parametrized functions from user input. User can explore the~function, move around it, make it very big to see parts of it in detail or make it small for convenient movement and rotate it.}
\item{Freedom of movement in room-scale, 3D environment, with six degrees of freedom. User can walk freely in a~room or use teleport as a~form of movement walk around the~virtual components of MathworldVR, explore the~world, parametrized function and components from different angles.}
\item{Ability to change and update parametrized function's variables and constraints through the~interactive settings panel. Function responds in real-time, giving the~user immediate feedback to see what impact the~change of variable has on parametrized function.}
\item{Ability to leverage VR hand controllers' interactions to grab and scale the~parametrized function. User's intuition expe}
\item{Virtual calculator-like 3D interface with clickable buttons to serve as a~panel for user input. User can choose from multiple variables, numbers from 0 to 9, operators and mathematical symbols to construct complex functions with It will include button for updating the~parametrized function, because user input needs to be parsed with Math.js library first.}
\end{itemize}

%\subsection{Virtual user interface}

