\section*{Appendix C}
\addcontentsline{toc}{section}{\numberline{}Appendix C}
\subsection*{Vytvorenie zoznamu skratiek a symbolov}

Ak sú v~práci skratky a symboly, vytvára sa \emph{Zoznam skratiek
a~symbolov} (a~ich dešifrovanie). V~prostredí \LaTeX{}u sa takýto
zoznam
ľahko vytvorí pomocou balíka \verb+nomencl+. Postup je nasledovný:
\begin{enumerate}
\item Do preambuly zapíšeme nasledujúce príkazy\\
\verb+\usepackage[slovak,noprefix]{nomencl}+\\ \verb+\makeglossary+
\item  V~mieste, kde má byť\/ vložený zoznam zapíšeme príkaz\\
\verb+\printglossary+
\item V miestach, kde sa vyskytujú skratky a symboly ich definíciu
zavedieme, napr. ako     	v~našom texte, príkazmi\\
\verb+\nomenclature{$\upmu$}{mikro, $10^{-6}$}+\\
\verb+\nomenclature{V}{volt, základná jednotka napätia v sústave SI}+\\
a dokument \uv{pre\LaTeX{}ujeme}.
\item Z~príkazového riadka spustíme program \verb+makeindex+
s~prepínačmi podľa použitého operačného systému, napr.~v~OS~GNU/Linux
s~distribúciou Ubuntu~$10.04$ a~verziou \verb+texlive 2009-7+
napíšeme:\\
\verb*+makeindex tukedip.glo -s nomencl.ist -o tukedip.gls+\\
~v~OS~Win\,XP s~verziou \verb+TeXLive 2010+
napíšeme:\\
\verb*+makeindex -o tukedip.gls -s nomencl.ist tukedip.glo+

\item Po opätovnom \uv{pre\LaTeX{}ovaní} dokumentu sa na
požadované
miesto vloží \emph{Zoznam skratiek a symbolov}.
\end{enumerate}
