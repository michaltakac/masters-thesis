\documentclass[]{tukethesis}
%% -------------------------------------------------------------------------
%% UTF-8 encoding used. Use pdflatex to compile your document
%% Tukethesis Class for Win XP and GNU/Linux
%% -------------------------------------------------------------------------
\usepackage[slovak,english]{babel}
\usepackage[utf8]{inputenc}
\usepackage[T1]{fontenc}
\usepackage{cmap}
\usepackage{lmodern}
\usepackage{dirtree}
%Define the listing package
\usepackage{listings} %code highlighter
\usepackage{xcolor} %use color
\definecolor{commentsGreen}{RGB}{95, 186, 125}
\definecolor{myFuchsia}{RGB}{175, 35, 143}
\definecolor{editorGray}{rgb}{0.95, 0.95, 0.95}
\definecolor{editorOcher}{rgb}{1, 0.5, 0} % #FF7F00 -> rgb(239, 169, 0)
\definecolor{editorGreen}{rgb}{0, 0.5, 0} % #007C00 -> rgb(0, 124, 0)
 
\lstdefinelanguage{JavaScript}{
  morekeywords={typeof, new, true, false, catch, function, return, null, catch, switch, var, if, in, while, do, else, default, case, break, import, export, default, const, let, module, module.exports, registerComponent, setObject3D, from, console.log, require},
  morecomment=[s]{/*}{*/},
  morecomment=[l]//,
  morestring=[b]",
  morestring=[b]',
  morestring=[b]`
}

\lstdefinelanguage{HTML5}{
        language=html,
        sensitive=true, 
            alsoletter={<>=-},
            otherkeywords={
            % HTML tags
            <, </, >,
            </a, <a, </a>,
            </abbr, <abbr, </abbr>,
            </address, <address, </address>,
            </area, <area, </area>,
            </area, <area, </area>,
            </article, <article, </article>,
            </aside, <aside, </aside>,
            </audio, <audio, </audio>,
            </audio, <audio, </audio>,
            </b, <b, </b>,
            </base, <base, </base>,
            </bdi, <bdi, </bdi>,
            </bdo, <bdo, </bdo>,
            </blockquote, <blockquote, </blockquote>,
            </body, <body, </body>,
            </br, <br, </br>,
            </button, <button, </button>,
            </canvas, <canvas, </canvas>,
            </caption, <caption, </caption>,
            </cite, <cite, </cite>,
            </code, <code, </code>,
            </col, <col, </col>,
            </colgroup, <colgroup, </colgroup>,
            </data, <data, </data>,
            </datalist, <datalist, </datalist>,
            </dd, <dd, </dd>,
            </del, <del, </del>,
            </details, <details, </details>,
            </dfn, <dfn, </dfn>,
            </div, <div, </div>,
            </dl, <dl, </dl>,
            </dt, <dt, </dt>,
            </em, <em, </em>,
            </embed, <embed, </embed>,
            </fieldset, <fieldset, </fieldset>,
            </figcaption, <figcaption, </figcaption>,
            </figure, <figure, </figure>,
            </footer, <footer, </footer>,
            </form, <form, </form>,
            </h1, <h1, </h1>,
            </h2, <h2, </h2>,
            </h3, <h3, </h3>,
            </h4, <h4, </h4>,
            </h5, <h5, </h5>,
            </h6, <h6, </h6>,
            </head, <head, </head>,
            </header, <header, </header>,
            </hr, <hr, </hr>,
            </html, <html, </html>,
            </i, <i, </i>,
            </iframe, <iframe, </iframe>,
            </img, <img, </img>,
            </input, <input, </input>,
            </ins, <ins, </ins>,
            </kbd, <kbd, </kbd>,
            </keygen, <keygen, </keygen>,
            </label, <label, </label>,
            </legend, <legend, </legend>,
            </li, <li, </li>,
            </link, <link, </link>,
            </main, <main, </main>,
            </map, <map, </map>,
            </mark, <mark, </mark>,
            </math, <math, </math>,
            </menu, <menu, </menu>,
            </menuitem, <menuitem, </menuitem>,
            </meta, <meta, </meta>,
            </meter, <meter, </meter>,
            </nav, <nav, </nav>,
            </noscript, <noscript, </noscript>,
            </object, <object, </object>,
            </ol, <ol, </ol>,
            </optgroup, <optgroup, </optgroup>,
            </option, <option, </option>,
            </output, <output, </output>,
            </p, <p, </p>,
            </param, <param, </param>,
            </pre, <pre, </pre>,
            </progress, <progress, </progress>,
            </q, <q, </q>,
            </rp, <rp, </rp>,
            </rt, <rt, </rt>,
            </ruby, <ruby, </ruby>,
            </s, <s, </s>,
            </samp, <samp, </samp>,
            </script, <script, </script>,
            </section, <section, </section>,
            </select, <select, </select>,
            </small, <small, </small>,
            </source, <source, </source>,
            </span, <span, </span>,
            </strong, <strong, </strong>,
            </style, <style, </style>,
            </summary, <summary, </summary>,
            </sup, <sup, </sup>,
            </svg, <svg, </svg>,
            </table, <table, </table>,
            </tbody, <tbody, </tbody>,
            </td, <td, </td>,
            </template, <template, </template>,
            </textarea, <textarea, </textarea>,
            </tfoot, <tfoot, </tfoot>,
            </th, <th, </th>,
            </thead, <thead, </thead>,
            </time, <time, </time>,
            </title, <title, </title>,
            </tr, <tr, </tr>,
            </track, <track, </track>,
            </u, <u, </u>,
            </ul, <ul, </ul>,
            </var, <var, </var>,
            </video, <video, </video>,
            </wbr, <wbr, </wbr>,
            />, <!
            }, 
        ndkeywords={
        % General
        =,
        % HTML attributes
        charset=, id=, width=, height=,
        % CSS properties
        border:, transform:, -moz-transform:, transition-duration:, transition-property:, transition-timing-function:
        },  
        morecomment=[s]{<!--}{-->},
        tag=[s]
}

%% ---- definicia slovenskych uvodzoviek
\chardef\clqq=18 \sfcode18=0
\chardef\crqq=16 \sfcode16=0
\def\uv#1{\clqq#1\crqq}
%% ------------------------------------
\renewcommand{\figurename}{Fig.}
\renewcommand{\tablename}{Tab.}
\renewcommand{\refname}{Bibliography}
\renewcommand{\listfigurename}{List of Figures}
\renewcommand{\listtablename}{List of Tables}
\renewcommand{\contentsname}{Contents}
%%
\usepackage{latexsym}
\usepackage{dcolumn} % alignment on a 'decimal point' in tabular mode
\usepackage{hhline}
\usepackage{amsmath}
\usepackage{nicefrac} % nice fractions
\usepackage{upgreek} % e.g. $\upmu\mathrm{m}$ type micrometer (mu)
\usepackage[final]{showkeys}%color%notref%notcite%final
\usepackage[noprefix]{nomencl}
\makeglossary % command to make *.glo file
\usepackage{parskip}%
%%
%\usepackage[dvips]{graphicx}
%\DeclareGraphicsExtensions{.eps, .mps}
\usepackage[pdftex]{graphicx}
\DeclareGraphicsExtensions{.pdf,.png,.jpg,.mps}
\graphicspath{{figures/}} % directory for figures
%%
%% numerical citations (vancouer style)
%\usepackage[numbers]{natbib}
%%
%% author-year citations (harvard style) -- prefered !!!
\usepackage{natbib} \citestyle{chicago}
% -----------------------------------------------------------------
%% tlač !!!
\usepackage[pdftex,unicode=true,bookmarksnumbered=true,
bookmarksopen=true,pdfmenubar=true,pdfview=Fit,linktocpage=true,
pageanchor=true,bookmarkstype=toc,pdfpagemode=UseOutlines,
pdfstartpage=1]{hyperref}
\hypersetup{%
baseurl={http://www.tuke.sk/sevcovic},
pdfcreator={pdfLaTeX},
pdfkeywords={Optimization, thesis, writing},
pdftitle={The Optimization of the Thesis Writing},
pdfauthor={J\'an Zelen\'y},
pdfsubject={Bachelor's Thesis}
} 
%% -----------------------------------------------------------------
%% START YOUR THESIS
%% -----------------------------------------------------------------
%%
%% PLEASE SELECT YOUR PREFERED THESIS TYPE
%%
%% A Bachelor's degree is a first degree at college or university
%%\bachelorthesis{Bachelor's Thesis}
%%
%% A Master's thesis is a second level college or university degree
\masterthesis{Master's Thesis}
%% -----------------------------------------------------------------
%% Ak praca nema 'podnazov' zakomentujte riadky \subtitle a \podnazov, 
%% alebo polozky nechajte prazdne
\author{Bc. Michal Takáč}
\title{Application of Virtual and Augmented Reality in Education}
\subtitle{}
\abstrakte{In this thesis we set to explore the possibilities of virtual and augmented reality (VR/AR) through the use of modern web technologies and development practices with the goal of finding novel approaches and applications in higher education with focus on mathematics. First a brief history of VR and AR is described followed by overview regarding the current problems and use cases of VR and AR systems and overview of current commercially available headsets. Then the novel, experimental Graphical User Interface (GUI) is presented, utilizing six degrees of freedom in room-scale virtual world within web page with use of interactions with VR hand controlelrs to provide strong visual representations and visualizations of parametrized functions, helping students to approach learning differently and understand difficult math concepts faster. In the end of thesis we propose future possibilities and how the technology could shape student's knowledge.}
\keywords{Virtual reality, Augmented reality, Education, Mathematics}
\degree{Engineer}
\university{Technical University of Košice}
\faculty{Faculty of Mining, Ecology, Process Control and Geotechnologies}
\facultyabbr{FBERG}
\department{Institute of Control and Informatization of Production Processes}
\departmentabbr{ÚRaIVP}
\fieldofstudy{Gathering and Processing of Earth Resources}
\studyprogramme{Informatization Process of Obtaining and Processing Raw Materials}
\supervisor{RNDr.~Andrea~Mojžišová, PhD.}
\consultanta{RNDr.~Jana~Pócsová, PhD.}
%\consultantb{RNDr.~Marián Čierny, DrSc.}
\dateofsubmission{April. 24. 2017} 
\town{Košice}
\abstrakt{Abstrakt je povinnou súčasťou každej práce. Je výstižnou
charakteristikou obsahu dokumentu. Nevyjadruje hodnotiace stanovisko
autora. Má byť\/ taký informatívny, ako to povoľuje podstata práce.
Text abstraktu sa píše ako jeden odstavec. Abstrakt neobsahuje odkazy
na samotný text práce. Mal by mať\/ rozsah 250 až 500 slov. Pri
štylizácii sa používajú celé vety, slovesá v činnom rode a tretej
osobe. Používa sa odborná terminológia, menej zvyčajné termíny,
skratky a~symboly sa pri prvom výskyte v texte definujú.}
\klucoveslova{Optimalizácia, záverečná práca, písanie}

\begin{document}
\renewcommand\theHfigure{\theHsection.\arabic{figure}}
\renewcommand\theHtable{\theHsection.\arabic{table}}
\bibliographystyle{dcu}
%% input the 'First page of the Thesis'
\firstpage

%% input the 'Title page of the Thesis'
\titlepage

%% input the 'Metadatasheet of the Thesis'
%\metadatasheet

%\errata % begin the 'Errata' 
%Ak je potrebné, autor na tomto mieste opraví chyby, ktoré našiel
%po vytlačení práce. Opravy sa uvádzajú takým písmom, akým je napísaná
%práca. Ak zistíme chyby až po vytlačení a zviazaní práce, napíšeme
%erráta na samostatný lístok, ktorý vložíme na toto miesto. Najlepšie je
%lístok prilepiť\/ \citep{kat}.
%
%Forma:
%
%\tabcolsep=10pt
%\begin{table}[!hb]
%	\centering
%	\begin{tabular}{|c|c|c|c|}\hline
%Strana & Riadok & Chybne & Správne \\\hline\hline
%12 & 6 & publikácia & prezentácia \\\hline
%22 & 23 & internet & intranet \\\hline
%& & & \\\hline
%& & & \\\hline
%	\end{tabular}
%\end{table}
%\kerrata

\abstrakte % Abstract in English

%\abstrakt % Abstract in Slovak

\endabstract % end of the Abstracts page

%% input the 'Assign of the Thesis'
\assignthesis

%% input the 'Declaration' of the author
\declaration
% I hereby declare that this thesis is my own work and effort. Where
% other sources of information have been used, they have been
% acknowledged.
%%
% Niektorí autori metodických príručiek o~záverečných prácach sa
% nazdávajú, že takéto vyhlásenie je zbytočné, nakoľko povinnosť
% vypracovať záverečnú prácu samostatne, vyplýva študentovi zo zákona
% a na  autora práce sa vzťahuje autorský zákon.

\acknowledgement % begin the 'Acknowledgement'
I would like to express my sincere thanks to RNDr~Andrea
Mojžišová, PhD, the main Supervisor, for her constant and constructive guidance throughout the study, valuable feedback and brainstorming sessions. Special mention should go
to RNDr~Jana Pócsová, PhD. for her interest in new approaches to teaching and her ability to listen and give feedback to novel ideas when others are not interested and also to prof. Steven Abbott for helping me out with Oculus Touch support. To all other who gave a hand, I say thank you
very much.
\endacknowledgement

\preface % begin the 'Preface'
Mathematical knowledge is often fundamental when solving real life problems. Especially, problems situated in the three-dimensional domain that require spatial skills are sometimes hard to understand for students. Many students have difficulties with spatial imagination and lack spatial abilities. Recently, a number of training studies have shown the usefulness of virtual reality in training spatial ability. Therefore I set myself a goal for finding inresections between virtual reality and higher education by building experimental virtual user interface(s) and testing them in educational environments, e.g. colleges and universities. In this thesis I'll focus on mathematics.
\endpreface

\thispagestyle{empty}
\tableofcontents
\newpage

\thispagestyle{empty}
%\addcontentsline{toc}{section}{\numberline{}List of Figures}
\listoffigures
\newpage

%\thispagestyle{empty}
%\addcontentsline{toc}{section}{\numberline{}List of Tables}
%\listoftables
%\newpage

\thispagestyle{empty}
%\addcontentsline{toc}{section}{\numberline{}List of Symbols and
%Abbreviations}
%\printglossary % input the 'List of Symbols and Abbreviations'
%\newpage

%\addcontentsline{toc}{section}{\numberline{}List of Therms}
\listofterms % begin the 'List of Therms'

\newpage
\lstlistoflistings

\begin{description}
%\item[DOF] Degrees of freedom
%	\item[Dizertácia] je rozsiahla vedecká rozprava, v~ktorej sa na
%základe vedeckého výskumu a~s~použitím (využitím) bohatého dokladového
%materiálu  ako i~vedeckých metód rieši zložitý odborný problém.
%	\item[Font] je súbor, obsahujúci predpisy na zobrazenie textu
%v~danom písme, napr. na tlačiarni. To čo vidíme je písmo; font je súbor
%a~nevidíme ho.
%	\item[Kritika] je odborne vyhrotený, prísny pohľad na hodnotenú
%vec. Medzi recenziou a kritikou je taký pomer ako medzi diskusiou a
%polemikou. Pri kritike treba prísnosť\/ chápať\/ v~tom zmysle, že sa
%v~nej okrem iného navrhuje, ako hodnotené dielo skvalitniť\/.
%	\item[Meter (m)] je vzdialenosť\/, ktorú svetlo vo vákuu prejde
%za časový interval $\nicefrac{1}{299\,792\,458}$~sekundy.
%	\item[Písmom] rozumieme vlastný vzhľad znakov.
%	\item[Problém] termín používaný vo všeobecnom zmysle vo vzťahu
%k~akejkoľvek duševnej aktivite, ktorá má nejaký rozoznateľný cieľ.
%Samotný cieľ nemusí byť\/ v~dohľadne. Problémy možno charakterizovať\/
%tromi rozmermi -- oblasťou, obtiažnosťou a veľkosťou.
%	\item[Proces] je postupnonosť\/ či rad časovo usporiadaných
%udalostí tak, že každá predchádzajúca udalosť\/ sa zúčastňuje na
%determinácii nasledujúcej udalosti.
\end{description}

\endlistofterms


%
\setcounter{page}{1}
\setcounter{equation}{0}
\setcounter{figure}{0}
\setcounter{table}{0}

\section*{Introduction}
\addcontentsline{toc}{section}{\numberline{}Introduction}
Mathematics is everywhere around us. For technical science it is an important part of the study. It is necessary to understand the context, discover it gradually and build the knowledge step by step. There is also a lot of literature that often approaches mathematics in the very human language. But still, many people are afraid of it, have prejudices and do not even try to master math. 

In this thesis we're introducing novel learning environment with the~use of virtual reality on the~web that can be used as possible way of teaching mathematics. It's an attempt to make studying mathematics more interesting and fun.

Our experimental project is called MathworldVR, which sets to explore the~possibilities and introduce novel methods of using web technologies for creating room-scale, immersive learning environment in virtual reality for helping students to explore, learn about and experiment with various parametrized functions. It’s also a~practical tool for teachers to showcase abstract concepts in concrete 3D environment during lectures.

Thesis starts with the~"Problem expression" section, where we describe our goal and methodology of how we're approaching the~development of MathworldVR. In "Analysis of current state" section, current state of virtual reality is described with introduction to the~technology, examples of different VR applications, what development platform and tools are used for their development and advantages of VR on the~web, called WebVR in short.

In "Technologies and tools for development", we provide detailed information about technologies used for building the~MathworldVR application. The~selection of technologies has it's meaning - all of them are used in modern web development by big companies and/or was created by them. All of them are also open-source, because MathworldVR will be open-sourced as well.

Design and functionality planning is explained in "Software design of virtual reality application" section, describing what decisions were made when architecting and designing MathworldVR and what functionality user can expect.

In "Implementation of virtual reality application", project setup, development process, testing and deployment is explained with detailed description of individual components that together makes the~MathworldVR experience possible.

Findings and future project roadmap is discussed in "Conclusion" section.
%
\section{Problem expression}
The main goal of master's thesis with the name "Application of Virtual and Augmented Reality in Education" is to conceptualize, design and implement a virtual learning environment (VLE) for higher education, primarily focused on mathematics. This environment will be available on the web in form of client-side, web application and accessible through desktop and mobile web browser. Application we'll be developing is called MathworldVR. Main emphasis will be put into desktop platform, since currently, it's the only platform that supports interactive VR capabilities of six degrees of freedom, head-mounted display (HMD) and hand controllers. We're specificaly focused on consumer version of HTC Vive (HMD) since we have one at our disposal and on which we'll be testing the implementation of application.

\subsection{Methodology of the thesiss}
Development of MathworldVR web application will include these steps:
\begin{itemize}
\item{At the beginning it's necessarry to become familiar with current state of VR support on the web and WebVR frameworks. This will require us to study the available WebVR API and it's state of implementation in modern, popular browsers.}
\item{After acknowledgement of limitations and possibilities of web browsers we can proceed to conceptualization and design. We need to define the functionality of MathworldVR, criteria for application and select the approach to development.}
\item{Then we can proceed to selecting a WebVR framework.}
\item{Before start of the development, we need select the technologies. MathworldVR will be client-side WebVR application and since WebVR is a novel technology, we need to look into modern development tools and approaches.}
\item{After selecting the development tools we need to prepare the development PC for work by installing the Node.js, NPM, initialization of the project with NPM and installing needed development tools and project dependencies in form of Node modules. For our needs, we'll go with code editor/IDE "VSCode" by Microsoft, which includes tools for efficient programming and static code analysis.}
\item{MathworldVR will be deployed to web server and tested manually in different web browsers that support WebVR API with use of HTC Vive HMD to ensure compatibility and good user experience.}
\item{In the end, MathworldVR is tested privately by students at Technical University in Košice and lecturers from the department of mathematics at FBERG.}
\end{itemize}

%
\include{theoreticalanalysis}
%
\section{Analysis of current state of virtual and augmented reality}

\subsection{Building virtual reality applications and experiences}

The leading platform for building VR experiences today is the game engine Unity, both because the company had the foresight to add support for the Oculus Rift development kit early on, but also simply because the early use cases from when Oculus Rift was still just a very successful Kickstarter project centered around video games. 

\subsection{Virtual reality on the web}

WebVR provides support for exposing virtual reality devices — for example head-mounted displays like the HTC Vive or Oculus Rift — to web apps, enabling developers to translate position and movement information from the display into movement around a 3D scene in browser. As of today, support for both head-mounted displays is available in experimental or development builds of Chrome and Firefox, with official release planned for 2017. This has numerous very interesting applications, from virtual product tours and interactive training apps to immersive first person games.
Unity, for instance, is able to make native builds for all major platforms from the same code base, including PC, Mac, Linux, iOS, Android and more. When made by professionals, such native builds will undoubtedly look better and run faster than a comparable VR experience built with WebGL and WebVR (at least AAA games or other experiences where high fidelity and performance are paramount).

The major advantage of WebVR over natively built experiences is the same as the web has always had over desktop apps and mobile apps today - no need to download and install anything. User just needs to click a link, type in a url, and the application runs directly in her browser. There’s no app store needed. Web developers can also take advantage of many open source libraries available on the internet.
%
\section{Conceptualisation and design}
At the~heart of user interface design is the~"user" and the~user is in the~"driver's seat". Therefeore it is critical that users of any virtual environment should be able to use the~interface intuitively, regardless of cultural diversities. the~goal of user interface design is to make the~user's interaction as simple and efficient as possible, in terms of accomplishing user goals. This is called user-centered design. \cite{badrul}

The~most important goal for designing virtual environment in MathworldVR is to reach learners more effectively through seamless integration of content and its organization, together with the~navigational and interactive controls that user use to work with the~content.


\subsection{Inputs and objections}
MathworldVR is designed as a~client-side, single page WebVR application. That means user doesn't have to install any additional software besides the~supported experimental browser (this will change in future and user's would not need to install any additional browsers since Mozilla Firefox will introduce full WebVR support in version 55 and Google Chrome should introduce WebVR support in version 59). To be able to use MathworldVR properly, user needs to have on of HTC Vive or Oculus Touch headset installed and prepared. It can be used also on any device through any browser supporting WebVR API in version 1.0 and up. If accessed through the~mobile device, user can look around the~3D environment but is not allowed to move from current position.

Code of the~application will be fully open-sourced under the~MIT license, hosted on \url{GitHub.com}, an~online version control repository. That will allow more skilled developers from around the~world contribute to the~project and provide easier working conditions with various universities.

When user opens MathworldVR website, she should be informed about supported browsers and provided with the~links to check her browser's WebVR support.

\subsection{Functionality planning}
MathworldVR will help students to explore various parametrized functions in 3D environment with use of VR hand controllers interactions. It will contain this functionality:

\begin{itemize}
\item{Visualization of various parametrized functions from user input. User can explore the~function, move around it, make it very big to see parts of it in detail or make it small for convenient movement and rotate it.}
\item{Freedom of movement in room-scale, 3D environment, with six degrees of freedom. User can walk freely in a~room or use teleport as a~form of movement walk around the~virtual components of MathworldVR, explore the~world, parametrized function and components from different angles.}
\item{Ability to change and update parametrized function's variables and constraints through the~interactive settings panel. Function responds in real-time, giving the~user immediate feedback to see what impact the~change of variable has on parametrized function.}
\item{Ability to leverage VR hand controllers' interactions to grab and scale the~parametrized function. User's intuition expe}
\item{Virtual calculator-like 3D interface with clickable buttons to serve as a~panel for user input. User can choose from multiple variables, numbers from 0 to 9, operators and mathematical symbols to construct complex functions with It will include button for updating the~parametrized function, because user input needs to be parsed with Math.js library first.}
\end{itemize}

%\subsection{Virtual user interface}


%
\section{Implementation of virtual reality application}
\lstset{%
    % Basic design
    backgroundcolor=\color{editorGray!50},
    basicstyle=\footnotesize\ttfamily\mdseries,   
    frame=l,
    % Line numbers
    xleftmargin={0.75cm},
    numbers=left,
    stepnumber=1,
    firstnumber=1,
    numberfirstline=true,
    % Code design   
    keywordstyle=\color{blue!100}\bfseries,
    commentstyle=\color{commentsGreen!100},
    ndkeywordstyle=\color{editorGreen!100},
    stringstyle=\color{myFuchsia!100},
    % Code
    language=HTML5,
    alsolanguage=JavaScript,
    alsodigit={.:;},
    tabsize=2,
    showtabs=false,
    showspaces=false,
    showstringspaces=false,
    extendedchars=true,
    breaklines=true,        
    % Support for German umlauts
    literate=%
    {Ö}{{\"O}}1
    {Ä}{{\"A}}1
    {Ü}{{\"U}}1
    {ß}{{\ss}}1
    {ü}{{\"u}}1
    {ä}{{\"a}}1
    {ö}{{\"o}}1
}

\subsection{Project initialization}
Before project can be initialized, Node.js has to be installed on PC used for development. To start a new JavaScript project, \texttt{npm init} command is used through command line, on Windows operating system it's usually PowerShell and on Unix systems, terminal. The multi-step wizard appears to guide developer through the project initialization process. It will generate the \texttt{package.json} file that includes these parts:

\begin{itemize}
\item{Name - title of the project}
\item{Version - project version, using semantic versioning}
\item{Description - short project description}
\item{Entry point - central file of our JavaScript code, usually index.js}
\item{Test command - NPM script for testing}
\item{Git repository - link to version control repository where application's code is hosted}
\item{Keywords - used to better allocate the project within NPM packages search}
\item{Author - project author}
\item{License - project license}
\end{itemize}

In the end, MathworldVR's \texttt{package.json} file looks like this:

\begin{lstlisting}
{
  "name": "mathworldvr",
  "version": "0.0.1",
  "description": "Math world in WebVR, powered by A-frame.",
  "main": "src/index.js",
  "scripts": {
    "test": "echo \"Error: no test specified\" && exit 1"
  },
  "repository": {
    "type": "git",
    "url": "git+https://github.com/michaltakac/mathworld.git"
  },
  "keywords": [
    "webvr",
    "math",
    "aframe",
    "mathworld",
    "vr",
    "room-scale"
  ],
  "author": "Michal Takáč",
  "license": "MIT",
  "bugs": {
    "url": "https://github.com/michaltakac/mathworld/issues"
  },
  "homepage": "https://github.com/michaltakac/mathworld#readme"
}
\end{lstlisting}

\subsection{Installing JavaScript packages}
In modern JavaScript development, applications are build by using many packages together. To install and use a package, developer first needs to search for one in NPM registry on \url{https://www.npmjs.com/}. Our application will need multiple packages. They are installed through command line with command \texttt{npm install <package-name>} found at the right side of every package detail web page:

\begin{figure}[ht!]
\centering
\includegraphics[width=0.9\textwidth]{npm-aframe}
\caption{Detail page of JavaScript package in NPM registry with the script to install highlighted.}
\label{r:61}
\end{figure}

To save a package into \texttt{package.json} file as a dependency, \texttt{--save} option has to be added to \texttt{npm install} command. To save it as a development dependency (only used for local development, doesn't get included in final build), we need to add \texttt{--save-dev} option instead. Packages with their respective version tag will be added to \texttt{package.json} file into specific location:

\begin{lstlisting}
"devDependencies": {
  "babel-cli": "^6.24.1",
  // ...
},
"dependencies": {
  "aframe": "^0.5.0",
  // ...
}
\end{lstlisting}

The list of used packages is included in System Manual.

\subsection{NPM scripts}
All of development, testing, bundling and deployment tasks are automated with \texttt{NPM scripts} defined in \texttt{package.json}.

\begin{lstlisting}
"scripts": {
  "start": "node server",
  "test": "jest",
  "test:watch": "npm test -- --watch",
  "coverage": "npm test -- --coverage && opn coverage/lcov-report/index.html",
  "lint": "eslint 'src/**/*.js' webpack.config.js server.js",
  "clean": "del 'build/!(.git*|Procfile)**'",
  "build:copy": "copyfiles -u 1 public/* public/**/* build",
  "build:clean": "rimraf \"build/!(.git*|Procfile)**\"",
  "prebuild": "npm run build:clean && npm run build:copy",
  "build": "cross-env NODE_ENV=production webpack"
},
\end{lstlisting}

\begin{itemize}
\item{"start" - Starts the development server.}
\item{"test" - Initialize Jest test runner that searches for all files that have \texttt{*.test.js} or \texttt{*.spec.js} in their file name and triggers all tests.}
\item{"test:watch" - Test runner will keep watching for file changes and will restart tests after every change.}
\item{"coverage" - Generates detailed test coverage report.}
\item{"lint" - Runs eslint for static code analysis testing according to preconfigured eslint rules in \texttt{.eslintrc}.}
\item{"clean" - Deletes the \texttt{/build} folder.}
\item{"build:copy" - Copies folders and files from \texttt{/public} into \texttt{/build}.}
\item{"build:clean" - Similar to "clean" command.}
\item{"prebuild" - Chain of NPM scripts that will be executed before actual "build" script.}
\item{"build" - Production-ready build that can be deployed to server.}
\end{itemize}

\subsection{Project structure}

\begin{figure}[ht!]
\centering
\includegraphics[width=0.3\textwidth]{folder-structure}
\caption{Project folder structure}
\label{r:62}
\end{figure}


\subsection{Webpack configuration}
Webpack is used as a tool to build JavaScript modules in MathworldVR application. It simplifies development workflow by quickly constructing a dependency graph of JavaScript application and bundling them in the right order. Our webpack configuration is defined in \texttt{webpack.config.js} and includes optimisations to the code like minification, obfuscation or splitting vendor/CSS/JavaScript code for production build. It also includes the configuration of Babel transpiler, which transpiles ES6/ES7 code to ES5, supported by all modern browsers.

\subsection{Webpack development server}
Development server configuration is defined in \texttt{server.js}. To start the server, \texttt{npm start} command is used from command line.

\begin{figure}[ht!]
\centering
\includegraphics[width=0.9\textwidth]{npm-start}
\caption{Started development server.}
\label{r:63}
\end{figure}

\subsection{Setting up Redux for application state management}
In Redux, all the application state is stored as a single object. We need to think of its shape before writing any code. In MathworldVR, there are 5 main "features" -  \texttt{calculator}, interactable \texttt{function box} with grid, inside which resides the \texttt{parametric function} handling 3D visualization and \texttt{settings} for this visualization, represented by interactive settings panel. Each of those has it's own state. These are combined together into one, global state, including additional \texttt{user} state that represents user position and \texttt{ui} state that represents traditional 2D user interface behaviour, like the visibility of information panel displayed after MathworldVR page is rendered in browser. For each feature and additional state, one Reducer function exists and by using reducer composition, all reducers are combined together into one, root reducer. Code for reducer composition is defined in \path{src/reducers/index.js}.

\begin{lstlisting}[caption={Reducers composition code.},captionpos=b]
import { combineReducers } from 'redux'
import calculator from './calculator'
import functionBox from './functionBox'
import parametricFunction from './parametricFunction'
import settings from './settings'
import ui from './ui'

const rootReducer = combineReducers({
  calculator,
  functionBox,
  parametricFunction,
  settings,
  ui
})

export default rootReducer
\end{lstlisting}

\texttt{combineReducers()} method generates a function that calls all reducers with the slices of state selected according to their keys, and combining their results into a single object again.

Each reducer is defined in it's own file that also includes the \texttt{initialState} object, which is passed into reducer function as a first argument, alongside action as a second argument. The reducer is a pure function that takes the previous state and an action, and returns the next state.

Actions are payloads of information that send data from application to Redux store. describe the fact that something happened, but don't specify how the application's state changes in response - this is the job of reducers. They are the only source of information for the store. To trigger an action, they must be dispatched using \texttt{store.dispatch()} method. Actions are plain JavaScript objects that must have a type property, indicating the type of action being performed. Types should typically be defined as string constants. Action creators are functions that create actions and returns them which makes them portable and easy to test. They are defined in \path{src/actions/index.js}.

The Store is the object that brings actions and reducers together. The store has the following responsibilities:

\begin{itemize}
\item{Holds application state.}
\item{Allows access to state via \texttt{getState()}.}
\item{Allows state to be updated via \texttt{dispatch(action)}.}
\item{Registers listeners via \texttt{subscribe(listener)}.}
\item{Handles unregistering of listeners via the function returned by \textttsubscribe(listener)}.}
\end{itemize}

It's important to note that Redux applications only have a single store. When we want to split the data handling logic, we'll use reducer composition instead of multiple stores.

Store configuration code for development is defined in \path{src/store/configureStore.dev.js}, which includes hot-reloading of reducers and action logger For production build, different store configuration is defined in \path{src/store/configureStore.prod.js}. Usage of correct configuration is determined by \texttt{NODE\char`_ENV} environment variable.

\begin{lstlisting}[caption={Loading store configuration dynamically.},captionpos=b]
if (process.env.NODE_ENV === 'production') {
  module.exports = require('./configureStore.prod')
} else {
  module.exports = require('./configureStore.dev')
}
\end{lstlisting}

\subsection{Combining A-Frame and React}
A-Frame framework is built on top of the DOM, thus web library such as React can be put cleanly on top of A-Frame. \cite{aframe-react}

A-Frame is an entity-component-system (ECS) framework exposed through HTML. ECS is a pattern used in game development that favors composability over inheritance, which is more naturally suited to 3D scenes where objects are built of complex appearance, behavior, and functionality. \cite{aframe-react}

In A-Frame, HTML attributes map to components which are composable modules that are plugged into \texttt{<a-entity>}s to attach appearance, behavior, and functionality. \cite{aframe-react}

MathworldVR is using \texttt{aframe-react} module as a thin layer on top of A-Frame to bridge with React. It passes React props directly to A-Frame using refs and \texttt{\justify .setAttribute()}, bypassing the DOM. This works since A-Frame's \texttt{\justify .setAttribute()} is able to take non-string data such as objects, arrays, or elements and synchronously modify underlying 3D scene graph. \cite{aframe-react}

With \texttt{aframe-react}, we get the the 3D and VR architecture of A-Frame, and the view and state ergonomics of React. React can be used to bind application and state data to the values of A-Frame components. And we still have access to all the features and performance of A-Frame as well as A-Frame's community component ecosystem. \cite{aframe-react}

\subsection{A-Frame scene}
3D scene initialization code is defined in \path{src/components/VRScene/index.js} by implementing the \texttt{<a-scene>} primitive from A-Frame API. It creates \texttt{THREE.Scene()} instance, sets the WebGL renderer by creating new \texttt{THREE.WebGLRenderer()} instance and combines them into a render loop. Abstractions like these makes the A-Frame framework a great WebVR development tool.

\begin{lstlisting}[caption={\textsl{VRScene} component.},captionpos=b]
export default class VRScene extends React.Component {
  render() {
    return (
      <a-scene>
        {this.props.children}
      </a-scene>
    )
  }
}
\end{lstlisting}

Additional A-Frame-specific libraries are loaded at the start of VRScene file:

\begin{lstlisting}
// A-frame Components by community
import 'aframe'
import 'aframe-teleport-controls'
import 'super-hands'
import physics from 'aframe-physics-system'

// Libraries used by MathworldVR (Three.js, A-Frame, etc.)
import 'lib'
\end{lstlisting}

Here, we're loading \texttt{aframe-physics-system} which has specific initialization requirement - we need to call a function \texttt{physics.registerAll()} from within VRcene after it's loaded. That means we need to put it inside \texttt{componentWillMount()} function, which is React's standard lifecycle method:

\begin{lstlisting}
componentWillMount() {
  // Initialize aframe-physics-system
  physics.registerAll()
}
\end{lstlisting}

To use \texttt{aframe-physics-system}, \texttt{physics="gravity: 0"} is added to \texttt{<a-scene>} primitive:

\begin{lstlisting}
<a-scene physics="gravity: 0">
  {this.props.children}
</a-scene>
\end{lstlisting}

\subsection{Camera component}
Camera component is implementing the \texttt{camera} from A-Frame API which is an abstraction over creation of a new \texttt{THREE.PerspectiveCamera} instance. Any additional properties are then passed into the main entity. \texttt{Camera} component is defined in \path{src/components/Camera/index.js}.

\begin{lstlisting}[caption={\textsl{Camera} component code.},captionpos=b]
import React from 'react'
import { Entity } from 'aframe-react'

const Camera = (props) => {
  return (
    <Entity camera="" {...props} />
  )
}

export default Camera
\end{lstlisting}

\subsection{Sky component}
Sky component is implementing the \texttt{geometry} and \texttt{material} from A-Frame API. It's used to achieve the feeling of a sky around the user. \texttt{geometry.primitive} property is set to \texttt{'sphere'} with \texttt{geometry.radius} of 30 meters and \texttt{sky.jpg} texture prepared for a 360-degree image viewer is added to \texttt{material.src} property, pointing to an URL where image is stored. Any additional properties are then passed into the main entity. \texttt{Sky} component is defined in \path{src/components/Sky/index.js}.

\begin{lstlisting}[caption={\textsl{Sky} component code.},captionpos=b]
import React from 'react'
import { Entity } from 'aframe-react'

const Sky = (props) => {
  return (
    <Entity
      geometry={{ primitive: 'sphere', radius: 30, phiLength: 360, phiStart: 0, thetaLength: 90 }}
      material={{ shader: 'flat', src: 'url(sky.jpg)', side: 'back', height: 2048, width: 2048 }}
      {...props}
    />
  )
}

export default Sky
\end{lstlisting}

\subsection{Plane component}
Plane component is implementing the \texttt{geometry} and \texttt{material} from A-Frame API. It's used as a floor under the user. \texttt{geometry.primitive} property is set to \texttt{'circle'} with \texttt{geometry.radius} of 12 meters and \texttt{floor.jpg} texture is added to \texttt{material.src} property, pointing to an URL where image is stored. Main entity is rotated 90-degrees around the X-axis. To accept light from Lights component, \texttt{material.shader} property is set to \texttt{'flat'}. Any additional properties are then passed into the main entity. \texttt{Plane} component is defined in \path{src/components/Plane/index.js}.

\begin{lstlisting}[caption={\textsl{Plane} component code.},captionpos=b]
import React from 'react'
import { Entity } from 'aframe-react'

const Plane = (props) => {
  return (
    <Entity
      geometry={{ primitive: 'circle', radius: 12 }}
      material={{ src: 'url(floor.jpg)', shader: 'flat', roughness: 0 }}
      rotation="-90 0 0"
      static-body
      {...props}
    />
  )
}

export default Plane
\end{lstlisting}

\subsection{Lights component}
\texttt{Lights} component is implementing the \texttt{light} from A-Frame API. It's used for lighting up the scene and main points of interest, such as FunctionBox, Calculator and SettingsPanel components. MathworldVR uses two types of light - \texttt{'point'} and \texttt{'hemisphere'}, grouped into single component. Light intensity is set with \texttt{intensity} property. \texttt{Lights} component is defined in \path{src/components/Lights/index.js}.

\begin{lstlisting}[caption={\textsl{Lights} component code.},captionpos=b]
import React from 'react'
import { Entity } from 'aframe-react'

const Lights = (props) => {
  return (
    <Entity {...props}>
      <Entity light={{ type: 'point', color: '#fff', intensity: 0.6 }} position={{ x: 3, y: 10, z: 1 }} />
      <Entity light={{ type: 'point', color: '#fff', intensity: 0.2 }} position={{ x: -3, y: -10, z: 1 }} />
      <Entity light={{ type: 'hemisphere,', groundColor: '#888', intensity: 0.8 }} />
    </Entity>
  )
}

export default Lights
\end{lstlisting}

\subsection{Text component}
\texttt{Text} component is implementing the \texttt{text} from A-Frame API. It's used for displaying 2D text inside 3D environment. \texttt{Text} component is defined in \path{src/components/Text/index.js}.

\begin{lstlisting}[caption={\textsl{Lights} component code.},captionpos=b]
import React from 'react'
import { Entity } from 'aframe-react'

const Text = ({ align, color, letterSpacing, lineHeight, opacity, value, width, zOffset, ...props }) => {
  return (
    <Entity
      text={{ value, width, align, letterSpacing, lineHeight, color, opacity, zOffset }}
      {...props}
    >
      {props.children}
    </Entity>
  )
}

export default Text
\end{lstlisting}

\subsection{Calculator component}
\subsubsection{Action types}
Calculator action types are defined in \path{src/actions/index.js} as string constants and exported individually to ensure they can be imported in other files and also tested.

\begin{lstlisting}[caption={\texttt{calculator} action types.},captionpos=b]
export const CALCULATOR_WRITE_TEXT = 'CALCULATOR_WRITE_TEXT'
export const CALCULATOR_BACKSPACE = 'CALCULATOR_BACKSPACE'
export const CALCULATOR_CLEAR_TEXT = 'CALCULATOR_CLEAR_TEXT'
\end{lstlisting}

\subsubsection{Action creators}
Calculator's action creators are pure functions that return payloads of information about type of \texttt{CALCULATOR} action being dispatched and, if needed, also the data we want to send to Redux store. They are defined in \path{src/actions/index.js}

\begin{lstlisting}[caption={Action for writing text to calculator display.},captionpos=b]
export const calculatorWriteText = (text) => ({
    type: CALCULATOR_WRITE_TEXT,
    text
})
\end{lstlisting}

\begin{lstlisting}[caption={Action to remove character from calculator display.},captionpos=b]
export const calculatorBackspace = () => ({
    type: CALCULATOR_BACKSPACE
})
\end{lstlisting}

\begin{lstlisting}[caption={Action to completely clear the calculator display.},captionpos=b]
export const calculatorClearText = () => ({
    type: CALCULATOR_CLEAR_TEXT
})
\end{lstlisting}

\subsubsection{Initial state}
Calculator component's initial state is defined in \path{src/reducers/calculator/index.js} as a \texttt{initialState} object.

\begin{lstlisting}[caption={Initial state of a \textsl{calculator}.},captionpos=b]
const initialState = {
  displayText: 'x^2 + y^2',
}
\end{lstlisting}

\subsubsection{Reducer function}
Reducer function for \texttt{Calculator} component is defined in \path{src/reducers/calculator/index.js}. From this file it's exported as a default function that takes state (with initialState as a default value) and action as its arguments and returns new state according to \texttt{CALCULATOR} actions. Possible \texttt{ActionTypes} are conditionally checked with JavaScript's \texttt{switch} statement for a better code readability.

\begin{lstlisting}[caption={Addition of text to \texttt{state.calculator.displayText}.},captionpos=b]
export default (state = initialState, action) => {
  switch (action.type) {
    // ...
    case ActionTypes.CALCULATOR_WRITE_TEXT:
      return { ...state, displayText: `${state.displayText}${action.text}` }
   // ...
    default: return state
  }
}
\end{lstlisting}

\begin{lstlisting}[caption={Remove one character from  \texttt{state.calculator.displayText}.},captionpos=b]
export default (state = initialState, action) => {
  switch (action.type) {
    // ...
    case ActionTypes.CALCULATOR_BACKSPACE:
      return { ...state, displayText: state.displayText.slice(0, -1) }
   // ...
    default: return state
  }
}
\end{lstlisting}

\begin{lstlisting}[caption={Clear the \texttt{state.calculator.displayText}.},captionpos=b]
export default (state = initialState, action) => {
  switch (action.type) {
    // ...
    case ActionTypes.CALCULATOR_CLEAR_TEXT:
      return { ...state, displayText: '' }
   // ...
    default: return state
  }
}
\end{lstlisting}

\subsubsection{Higher-order component}
State and actions that Calculator component needs to function properly are mapped to properties in Calculator container, also called "higher-order component", because it's "aware" of application's state that is mapped into it. Code for Calculator higher-order component creation is defined in \path{src/containers/Calculator/index.js}.

\begin{lstlisting}[caption={Function to map \texttt{calculator} state to component properties.},captionpos=b]
const mapStateToProps = (state) => ({
  displayText: state.calculator.displayText,
})
\end{lstlisting}

\begin{lstlisting}[caption={Function to map dispatchable \texttt{calculator} action creators to component properties.},captionpos=b]
const mapDispatchToProps = (dispatch) => ({
  writeText: (text) => dispatch(calculatorWriteText(text)),
  backspace: () => dispatch(calculatorBackspace()),
  clearText: () => dispatch(calculatorClearText()),
  updateEquation: (equation) => dispatch(parametricFunctionSetEquation(equation)),
})
\end{lstlisting}

\begin{lstlisting}[caption={Creation of \texttt{Calculator} higher-order component.},captionpos=b]
import { connect } from 'react-redux'
import { Calculator } from 'components'
// ...
export default connect(mapStateToProps, mapDispatchToProps)(Calculator)
\end{lstlisting}

\subsubsection{Presentational component}
Presentational Calculator component is implementing the \texttt{geometry} and \texttt{material} from A-Frame API. It's shape is determined by \texttt{geometry.primitive} which is set to \texttt{'box'} with width of \texttt{0.88} meters, height of \texttt{0.65} meters and depth of \texttt{0.01} meters. By default it's not moveable to ensure its position central to the MathworldVR experience. Properties passed into the component as function attributes includes state and dispatchable action that was mapped in it's higher-order component. \texttt{Calculator} code is defined in \path{src/components/Calculator/index.js}. Code preview displayed here doesn't include the CalcButton components, but they're utilizing the \texttt{writeText}, \texttt{backspace} and \texttt{clearText} actions. They are explained in next section.

\begin{lstlisting}[caption={Presentational \textsl{Calculator} component code.},captionpos=b]
const Calculator = ({ displayText, writeText, backspace, clearText, updateEquation }) => {
  return (
    <Entity
      geometry="primitive: box; width: 0.88; height: 0.65; depth: 0.01;"
      material="shader: flat; side: double; color: #8d8547;"
    >
      { /* Calculator display */ }
      <Text value={displayText} />
      { /* --- BUTTONS --- */ }
      { /* ... */ }
    </Entity>
  )
}
      
\end{lstlisting}

\subsection{CalcButton component}
\texttt{CalcButton} is interactable component that is reacting to VR hand controllers interactions. When hand controller intersects with CalcButton in 3D environment, it fires an \texttt{hover-start} event, which starts the chain reaction of another two events by calling a \texttt{startInteraction()} method that fires Redux action passed to CalcButton as a property attribute and at the same time changes the depth and opacity of CalcButton. This is providing a visual cue - simulation of an actual click of a button on real calculator. After hand controller leaves the area of CalcButton, \texttt{hover-end} event is fired, initializing \texttt{endInteraction()} method which sets the depth and opacity back to default values. \texttt{CalcButton} component is defined in \path{src/components/CalcButton/index.js}.

\begin{figure}[ht!]
\centering
\includegraphics[width=0.7\textwidth]{calculator}
\caption{\textsl{Calculator} with multiple \textsl{CalcButton} components.}
\label{r:64}
\end{figure}

\subsection{ParametrizedFunction component}
First, the A-Frame API had to be extended with \texttt{parametricfunction} component:

\begin{lstlisting}[caption={Registering new A-Frame component \texttt{'parametricfunction'}.},captionpos=b]
AFRAME.registerComponent('parametricfunction', {
  schema: {/* ... */}, 
  init: function() {/* ... */},
  update: function() {/* ... */},
  remove: function() {/* ... */},
})
\end{lstlisting}

A-Frame components should have schema defined to ensure what data is passed into them. \texttt{parametricfunction}'s schema includes \texttt{equation}, \texttt{segments}, \texttt{xMin}, \texttt{xMax}, \texttt{yMin}, \texttt{yMax}, \texttt{zMin}, \texttt{zMax} and  \texttt{functionColor}, with their respective default values.

\begin{lstlisting}[caption={\texttt{parametricfunction} A-Frame component schema with default values.},captionpos=b]
schema: {
  equation: { type: 'string', default: '' },
  segments: { type: 'number', default: 20 },
  xMin: { type: 'number', default: -5 },
  xMax: { type: 'number', default: 5 },
  yMin: { type: 'number', default: -5 },
  yMax: { type: 'number', default: 5 },
  zMin: { type: 'number', default: -5 },
  zMax: { type: 'number', default: 5 },
  functionColor: { type: 'string', default: '#bada55' }
}
\end{lstlisting}

Main functions of \texttt{parametricfunction} A-Frame component are:
\begin{itemize}
\item{Parsing the equation passed into it with use of \texttt{Math.js} module and visualizing it in 3D environment, inside FunctionBox component that provides the grid.}
\item{Updating the 3D visualization after settings in \texttt{SettingsPanel} changes.}
\end{itemize}

\begin{lstlisting}[caption={Parsing of the equation.},captionpos=b]
var equation = 'f(x,y) = ' + this.data.equation;
var parser = math.parser();

try {
  parser.eval(equation);
} catch (error) {
  return;
}
\end{lstlisting}

To get the parsed value as a function, \texttt{parser.get('f')} method is called:

\begin{lstlisting}
const f1 = parser.get('f');
parser.clear();
\end{lstlisting}

The code of \texttt{parametricfunction} A-Frame component implementation is defined in \path{src/lib/components/parametricfunction.js}
    
\subsubsection{Action types}
ParametricFunction action types are defined in \path{src/actions/index.js} as string constants and exported individually to ensure they can be imported in other files and also tested.

\begin{lstlisting}[caption={\texttt{parametricFunction} action types.},captionpos=b]
export const PARAMETRIC_FUNCTION_SET_EQUATION = 'PARAMETRIC_FUNCTION_SET_EQUATION'
\end{lstlisting}

\subsubsection{Action creators}
ParametricFunction's action creators are pure functions that return payloads of information about type of \texttt{PARAMETRIC\char`_FUNCTION} action being dispatched and, if needed, also the data we want to send to Redux store. They are defined in \path{src/actions/index.js}

\begin{lstlisting}[caption={Action for setting the function for 3D visualization.},captionpos=b]
export const parametricFunctionSetEquation = (equation) => ({
  type: PARAMETRIC_FUNCTION_SET_EQUATION,
  equation,
})
\end{lstlisting}

\subsubsection{Initial state}
ParametricFunction component's initial state is defined inside the  \path{src/reducers/parametrizedFunction/index.js} file as a \texttt{initialState} object.

\begin{lstlisting}[caption={Initial state of a \textsl{calculator}.},captionpos=b]
const initialState = {
  equation: 'x^2 + y^2',
}
\end{lstlisting}

\subsubsection{Reducer function}
Reducer function for \texttt{ParametricFunction} component is defined in \path{src/reducers/parametricFunction/index.js}. From this file it's exported as a default function that takes state (with initialState as a default value) and action as its arguments and returns new state according to \texttt{PARAMETRIC\char`_FUNCTION} actions. Possible \texttt{ActionTypes} are conditionally checked with JavaScript's \texttt{switch} statement for a better code readability.

\begin{lstlisting}[caption={Update the \texttt{state.parametricFunction.equation} value.},captionpos=b]
export default (state = initialState, action) => {
  switch (action.type) {
    // ...
    case ActionTypes.PARAMETRIC_FUNCTION_SET_EQUATION:
      return { ...state, equation: action.equation }
   // ...
    default: return state
  }
}
\end{lstlisting}

\subsubsection{Higher-order component}
State and actions that ParametricFunction component needs to function properly are mapped to properties in ParametricFunction container - higher-order component. Code for ParametricFunction higher-order component creation is defined in \path{src/containers/ParametricFunction/index.js}.

\begin{lstlisting}[caption={Function to map \texttt{parametricFunction}  and \texttt{settings} state to component properties.},captionpos=b]
const mapStateToProps = (state) => ({
  equation: state.parametricFunction.equation,
  segments: state.settings.segments,
  xMin: state.settings.xMin,
  xMax: state.settings.xMax,
  yMin: state.settings.yMin,
  yMax: state.settings.yMax,
  zMin: state.settings.zMin,
  zMax: state.settings.zMax,
  functionColor: state.settings.functionColor,
})
\end{lstlisting}

\begin{lstlisting}[caption={Creation of \texttt{ParametricFunction} higher-order component.},captionpos=b]
import { connect } from 'react-redux'
import { ParametricFunction } from 'components'
// ...
export default connect(mapStateToProps)(ParametricFunction)
\end{lstlisting}

\subsubsection{Presentational component}
Presentational ParametricFunction component is implementing the \texttt{\justify parametricfunction} from A-Frame API we extended. Properties passed into the component as function attributes includes state and dispatchable action that was mapped in it's higher-order component. \texttt{ParametricFunction} code is defined in \path{src/components/ParametricFunction/index.js}.

\begin{lstlisting}[caption={Presentational \texttt{ParametricFunction} component code.},captionpos=b]
const ParametricFuntion = ({ equation, segments, xMin, xMax, yMin, yMax, zMin, zMax, functionColor }) => {
  return (
    <Entity
      id="function-mesh"
      parametricfunction={{
        equation,
        segments,
        xMin,
        xMax,
        yMin,
        yMax,
        zMin,
        zMax,
        functionColor,
      }}
      grid="size: 2; step: 20"
    />
  )
}
\end{lstlisting}

\subsection{FunctionBox component}
\subsubsection{Action types}
FunctionBox action types are defined in \path{src/actions/index.js} as string constants and exported individually to ensure they can be imported in other files and also tested.

\begin{lstlisting}[caption={\texttt{functionBox} action types.},captionpos=b]
export const FUNCTION_BOX_SET_POSITION = 'FUNCTION_BOX_SET_POSITION'
\end{lstlisting}

\subsubsection{Action creators}
FunctionBox's action creators are pure functions that return payloads of information about type of \texttt{FUNCTION\char`_BOX} action being dispatched and, if needed, also the data we want to send to Redux store. They are defined in \path{src/actions/index.js}

\begin{lstlisting}[caption={Action for setting function box position.},captionpos=b]
export const functionBoxSetPosition = (position) => ({
  type: FUNCTION_BOX_SET_POSITION,
  position,
})
\end{lstlisting}

\subsubsection{Initial state}
\texttt{FunctionBox} component's initial state is defined in \path{src/reducers/functionBox/index.js} as a \texttt{initialState} object.

\begin{lstlisting}[caption={Initial state of a \textsl{functionBox}.},captionpos=b]
const initialState = {
  position: { x: 0.65, y: 1.45, z: -1.03 },
}
\end{lstlisting}

\subsubsection{Reducer function}
Reducer function for \texttt{FunctionBox} component is defined in \path{src/reducers/functionBox/index.js}. From this file it's exported as a default function that takes state (with \texttt{initialState} as a default value) and \texttt{action} as its arguments and returns new state according to \texttt{FUNCTION\char`_BOX} actions. Possible \texttt{ActionTypes} are conditionally checked with JavaScript's \texttt{switch} statement for a better code readability.

\begin{lstlisting}[caption={Updating the   \texttt{state.functionBox.position} value.},captionpos=b]
export default (state = initialState, action) => {
  switch (action.type) {
    case ActionTypes.FUNCTION_BOX_SET_POSITION:
      return { ...state, position: action.position }
    default: return state
  }
}
\end{lstlisting}

\subsubsection{Higher-order component}
State and actions that \texttt{FunctionBox} component needs to function properly are mapped to properties in Calculator container -  higher-order component. Code for FunctionBox higher-order component creation is defined in \path{src/containers/FunctionBox/index.js}.

\begin{lstlisting}[caption={Function to map \texttt{functionBox} state to component properties.},captionpos=b]
const mapStateToProps = (state) => ({
  position: state.functionBox.position,
})
\end{lstlisting}

\begin{lstlisting}[caption={Creation of \texttt{FunctionBox} higher-order component.},captionpos=b]
import { connect } from 'react-redux'
import { FunctionBox } from 'components'
// ...
export default connect(mapStateToProps, null)(FunctionBox)
\end{lstlisting}

\subsubsection{Presentational component}
Presentational FunctionBox component is implementing the \texttt{geometry} and \texttt{material} from A-Frame API. It's shape is determined by \texttt{geometry.primitive} which is set to \texttt{'box'} with width of \texttt{4} meters, height of \texttt{4} meters and depth of \texttt{4} meters. By default it's scaled down by 80\%, but since this component is interactive, it can be stretched with hand controller interactions. It can also be moved around the 3D environment. \texttt{Position} property is passed into the component from higher-order component. \texttt{FunctionBox} code is defined in \path{src/components/FunctionBox/index.js}.

\begin{lstlisting}[caption={Presentational \textsl{FunctionBox} component code.},captionpos=b]
const FunctionBox = ({ position, ...props }) => (
  <Entity
    id="function-box"
    className="interactive"
    geometry="primitive: box; width: 4; height: 4; depth: 4;"
    material="transparent: true; opacity: 0; shader: standard"
    scale="0.2 0.2 0.2"
    position={position}
    grabbable
    stretchable
    dynamic-body
    stop-flying
  >
    { props.children }
  </Entity>
)  
\end{lstlisting}

\subsection{SettingsPanel component}
First, the A-Frame API has to be extended with \texttt{datgui} component. Inside its \texttt{init} method, the \texttt{dat.guiVR} instance is created and injected into the component element property:

\begin{lstlisting}
const gui = dat.GUIVR.create( this.data.name );
this.el.setObject3D('gui', gui );
this.el.gui = gui;
\end{lstlisting}

To allow VR hand controllers interactions, \texttt{bindInput} method is used for creation of event listeners for \texttt{triggerdown}, \texttt{triggerup}, \texttt{gripdown} and \texttt{gripup} events:

\begin{lstlisting}
function bindInput(el, input) {
  el.addEventListener('triggerdown', function() {
    input.pressed(true);
  });
  el.addEventListener('triggerup', function() {
    input.pressed(false);
  });
  el.addEventListener('gripdown', function() {
    input.gripped(true);
  });
  el.addEventListener('gripup', function() {
    input.gripped(false);
  });
}
\end{lstlisting}

\subsubsection{Action types}
SettingsPanel action types are defined in \path{src/actions/index.js} as string constants and exported individually to ensure they can be imported in other files and also tested.

\begin{lstlisting}[caption={\texttt{settings} action types.},captionpos=b]
export const SETTINGS_SET_X_MIN = 'SETTINGS_SET_X_MIN'
export const SETTINGS_SET_Y_MIN = 'SETTINGS_SET_Y_MIN'
export const SETTINGS_SET_Z_MIN = 'SETTINGS_SET_Z_MIN'
export const SETTINGS_SET_X_MAX = 'SETTINGS_SET_X_MAX'
export const SETTINGS_SET_Y_MAX = 'SETTINGS_SET_Y_MAX'
export const SETTINGS_SET_Z_MAX = 'SETTINGS_SET_Z_MAX'
export const SETTINGS_SET_SEGMENTS = 'SETTINGS_SET_SEGMENTS'
export const SETTINGS_SET_FUNCTION_COLOR = 'SETTINGS_SET_FUNCTION_COLOR'
\end{lstlisting}

\subsubsection{Action creators}
SettingsPanel's action creators are pure functions that return payloads of information about type of \texttt{SETTINGS} action being dispatched and, if needed, also the data we want to send to Redux store. They are defined in \path{src/actions/index.js}

\begin{lstlisting}[caption={Action for setting fnction segments.},captionpos=b]
export const settingsSetSegments = (segments) => ({
  type: SETTINGS_SET_SEGMENTS,
  segments,
})
\end{lstlisting}

\subsubsection{Initial state}
SettingsPanel component's initial state is defined in \path{src/reducers/settings/index.js} as a \texttt{initialState} object.

\begin{lstlisting}[caption={Initial state of a \textsl{calculator}.},captionpos=b]
const initialState = {
  xMin: -1,
  yMin: -1,
  zMin: -4,
  xMax: 1,
  yMax: 1,
  zMax: 4,
  segments: 30,
  functionColor: '#bada55',
}
\end{lstlisting}

\subsubsection{Reducer function}
Reducer function for \texttt{SettingsPanel} component is defined in \path{src/reducers/settings/index.js}. From this file it's exported as a default function that takes state (with initialState as a default value) and action as its arguments and returns new state according to \texttt{SETTINGS} actions. Possible \texttt{ActionTypes} are conditionally checked with JavaScript's \texttt{switch} statement for a better code readability.

\begin{lstlisting}[caption={Update of \texttt{state.settings.segments} value.},captionpos=b]
export default (state = initialState, action) => {
  switch (action.type) {
    // ...
    case ActionTypes.SETTINGS_SET_SEGMENTS:
      return { ...state, segments: action.segments }
   // ...
    default: return state
  }
}
\end{lstlisting}

\subsubsection{Higher-order component}
State and actions that SettingsPanel component needs to function properly are mapped to properties in SettingsPanel container - higher-order component. Code for SettingsPanel higher-order component creation is defined in \path{src/containers/SettingsPanel/index.js}.

\begin{lstlisting}[caption={Function to map dispatchable \texttt{settings} action creators to component properties.},captionpos=b]
const mapDispatchToProps = (dispatch) => {
  return {
    setXMin: (xMin) => dispatch(settingsSetXMin(xMin)),
    setYMin: (yMin) => dispatch(settingsSetYMin(yMin)),
    setZMin: (zMin) => dispatch(settingsSetZMin(zMin)),
    setXMax: (xMax) => dispatch(settingsSetXMax(xMax)),
    setYMax: (yMax) => dispatch(settingsSetYMax(yMax)),
    setZMax: (zMax) => dispatch(settingsSetZMax(zMax)),
    setSegments: (segments) => dispatch(settingsSetSegments(segments)),
    setFunctionColor: (color) => dispatch(settingsSetFunctionColor(color)),
  }
}
\end{lstlisting}

\begin{lstlisting}[caption={Creation of \texttt{SettingsPanel} higher-order component.},captionpos=b]
import { connect } from 'react-redux'
import { SettingsPanel } from 'components'
// ...
export default connect(null, mapDispatchToProps)(SettingsPanel)
\end{lstlisting}

\subsubsection{Presentational component}
Presentational SettingsPanel component is implementing the \texttt{datgui} from A-Frame API that we extended. Individual settings are handled by \texttt{SettingsController} component, which triggers the action when user changes the value by pointing to the settings with VR hand controller and pressing the trigger button. \texttt{SettingsPanel} code is defined in \path{src/components/SettingsPanel/index.js}.

\begin{lstlisting}[caption={Presentational \textsl{Calculator} component code.},captionpos=b]
const SettingsPanel = ({
  name,
  controllerLeft,
  controllerRight,
  setSegments,
  ...props
}) => {
  return (
    <Entity datgui={{ name, controllerLeft, controllerRight }} {...props}>
      <SettingsController type="slider" name="segments" step={1} min={1} max={50} initialState={30} actionToTrigger={setSegments} />
      { /* ...other settings... */ }
    </Entity>
  )
} 
\end{lstlisting}

\begin{figure}[ht!]
\centering
\includegraphics[width=0.8\textwidth]{function_settings}
\caption{\texttt{SettingsPanel} component}
\label{r:65}
\end{figure}

\newpage
\subsection{Hand controller components}
VR hand controllers implement the \texttt{super-hands} A-Frame API that was extended by importing the \texttt{aframe-super-hands-component} in MathworldVR's \texttt{VRScene} component. \texttt{super-hands} A-Frame component adds natural, intuitive hand controller interactions, interprets input from tracked controllers and collision detection into interaction gestures and communicates those gestures to target entities for them to respond.

The currently implemented gestures are:
\begin{itemize}
\item{Hover - Holding a controller in the collision space of an entity.}
\item{Grab - Pressing a button while hovering an entity, potentially also moving it.}
\item{Stretch - Grabbing an entity with two hands and resizing.}
\item{Drag-drop - Dragging an entity onto another entity.}
\end{itemize}

\texttt{super-hand} includes components for typical reactions to the implemented gestures: \texttt{hoverable}, \texttt{grabbable}, \texttt{stretchable}, and \texttt{drag-droppable}. Code for hand controllers implementation used in MathworldVR is defined in \path{src/components/LeftController/index.js} and \path{src/components/LeftController/index.js} respectively.

\begin{figure}[ht!]
\centering
\includegraphics[width=0.7\textwidth]{grab}
\caption{Grabbing functionality}
\label{r:5}
\end{figure}

\begin{figure}[ht!]
\centering
\includegraphics[width=0.7\textwidth]{scaling}
\caption{Scaling functionality}
\label{r:6}
\end{figure}

\newpage
\subsection{AttentionBox component}
\texttt{AttentionBox} component is used for displaying semi-transparent rectangle with MathworldVR project information in the middle of the browser screen. \texttt{AttentionBox} component is defined in \path{src/components/AttentionBox/index.js}.

\subsection{Adding components into A-Frame scene}
To add MathworldVR components into the 3D environment, they need to be included as a children components of \texttt{VRScene} component:

\begin{lstlisting}
<VRScene>
  <AttentionBox />
  <LeftController />
  <RightController />

  <FunctionBox>
    <ParametricFunction />
  </FunctionBox>

  <Calculator />

  <SettingsPanel
    name="Function settings"
    position={{ x: -0.37, y: 1.93, z: -0.34 }}
    rotation={{ x: 10, y: 30, z: 0 }}
    scale={{ x: 0.5, y: 0.5, z: 0.5 }}
  />

  <Sky />
  <Lights />
  <Plane />
</VRScene>
\end{lstlisting}

\subsection{Building and deploying the application to web hosting}
Application's JavaScript code is bundled together by using NPM script \texttt{npm run build}, which produces production-ready bundle. Build script triggers pre-build task, which gets executed before the build and consists of  \texttt{npm run build:clean} (for removing all files from \texttt{/build} folder) and \texttt{npm run build:copy} (for copying static files like \texttt{index.html}, assets, images, CSS stylesheets into cleaned \texttt{/build} folder) scripts, called synchronously.

To deploy the application, \texttt{/build} folder contents are copied to the server, hosted on \url{https://www.fortrabbit.com/}. Hosting is sponsored by \textsl{Sleighdogs, GmbH} company, which author is working for during writing of this thesis. Server us hosted under \url{http://vr.sld.gs/mathworldvr/} domain.

\newpage

\begin{figure}[ht!]
\centering
\includegraphics[width=1\textwidth]{main}
\caption{Finalized MathworldVR application, deployed on live server.}
\label{r:69}
\end{figure}


%
\section{Conclusion}
Goal of the thesis was to explore possibilities of virtual reality on the web, design and implement VR application called MathworldVR that could help students explore, experiment and learn about various parametrized functions, allowing them to build an intuition of how such functions behave according to changes of their variables.

We managed to not only successfuly implement the WebVR application, but also push forward the whole ecosystem of web-based virtual reality provided by A-Frame framework. We decided to open-source MathworldVR so more ideas can "flow" freely into the project through contributors from all around the world.

Many components were created from scratch for the project and we also inspired other A-Frame developers by sharing the progress along the way on social media websites Facebook, Twitter and Instagram.

MathworldVR cought an eye of prof. Steven Abbott (\url{https://www.stevenabbott.co.uk/}), former Visiting Professor at the School of Mechanical Engineering at University Leeds, who helped us with the Oculus Touch support.

We also joined Virtuleap's online, worldwide WebVR hackathon (\url{http://www.virtuleap.com/}), where we managed to move into finals from 33 participanting projects and ended in 10th place overall. This triggered the public interest, helping get MathworldVR mentioned in A-Frame weekly blog and VentureBeat, \textsl{"the leading source for news, events, groundbreaking research and perspective on technology innovation"} (\url{https://venturebeat.com/2017/01/10/virtuleap-hackathon-generates-a-bunch-of-webvr-projects/}).

Current functionality of MathworldVR is very limited, but interest and recognition this project received shown that it's definitely worth to take it to the next level in near future. This will be also easier since we open-sourced it. 

Next step will be building the infrastructure, adding the multi-user support and creating a back-end with database. We're looking into Elixir programming language and Phoenix framework, because it looks very promising. It leverages the ability of Erlang's virtual machine to handle millions of connections alongside Elixir's beautiful syntax and productive tooling for building fault-tolerant systems. This will make creating different (public or private) "rooms" and persisting the position, rotation and application state, possible.

Another planned features include: more examples in form of "mathematical rooms" of different kind, better calculator component, intuitive menu added to VR hand controller incorporating multiple options and layers, 2D graphs support, double and surface integrals, vectors and many more. We plan to create a public roadmap of planned features to serve as a guide for developers, universities and others who are interested about the progress.
%
%%
\begin{thebibliography}{1}
\addcontentsline{toc}{section}{\numberline{}{Bibliography}}

\harvarditem{Badrul}{2011}{badrul}
BADRUL, K. 2011. \emph{User Interface Design for Virtual Environments: Challenges and Advances.} USA: IGI Global, 2011. 375~s. ISBN 9781613505175

\harvarditem{Flanagan}{2006}{flanagan}
FLANAGAN, D. 2006. \emph{JavaScript: The Definitive Guide (5th edition).} USA: O'Reilly, 2006. 1032~s. ISBN 978-0-596-10199-2

\harvarditem{Narayan}{2015}{narayan}
NARAYAN, P. 2015. \emph{Learning ECMAScript 6.} USA: Packt Publishing, 2015. 202~s. ISBN 9781785884443

\harvarditem{Loeffler}{1995}{loeffler}
LOEFFLER, C. 1995. \emph{Distributed Virtual Reality: Applications for Education, Entertainment and Industry.} \url{http://www.wiumlie.no/1993/telektronikk-4-93/Loeffler_C_E.html}

\harvarditem{Kaufmann}{2011}{kaufmann}
KAUFMANN, H. 2011. \emph{Virtual Environments for Mathematics and Geometry Education.} In:~Themes In Science and Technology Education. Vienna~: Klidarithmos Computer Books, 2011, Special Issue, pp. K~131--152

\harvarditem{NPM}{2017}{npm}
\emph{NPM.} [online]. Retrieved April 2, 2017 from \url{https://docs.npmjs.com/all}

\harvarditem{A-Frame}{2017}{aframe-intro}
\emph{A-Frame.} [online]. Retrieved April 2, 2017 from \url{https://aframe.io/docs/0.5.0/introduction/}

\harvarditem{Math.js}{2017}{mathjs}
\emph{Math.js.} [online]. Retrieved April 2, 2017 from \url{http://mathjs.org/}

\harvarditem{Gackenheimer}{2015}{gackenheimer}
GACKENHEIMER, C. 2015. \emph{Introduction to React.} USA: Apress, 2015 129~s. ISBN 978-1-4842-1245-5

\harvarditem{Facebook, Inc.}{2015}{flux}
FACEBOOK, Inc. 2015. \emph{Flux - In Depth Overview.} [online]. Retrieved April 13, 2017 from \url{https://facebook.github.io/flux/docs/in-depth-overview.html}

\harvarditem{Redux.org}{2017}{redux-intro}
\emph{Redux.} [online]. Retrieved April 13, 2017 from \url{https://aframe.io/docs/0.5.0/introduction/}

\harvarditem{aframe-react}{2017}{aframe-react}
\emph{aframe-react.} [Online]. Retrieved April 13, 2017 from \url{https://github.com/aframevr/aframe-react/blob/master/README.md}

\harvarditem{thesixthaxis.com}{2017}{thesixthaxis.com}
\emph{Job Simulator revenue.} [Online]. Retrieved April 17, 2017 from \url{http://www.thesixthaxis.com/2017/01/09/job-simulators-3-million-in-sales-show-that-vr-can-be-profitable/}

\harvarditem{Unreal Engine}{2017}{unrealengine}
\emph{Unreal Engine 4.} [Online]. Retrieved April 17, 2017 from \url{https://www.unrealengine.com/unreal-engine-4}

\harvarditem{HTC Vive User Guide}{2017}{htc-vive-user-guide}
\emph{HTC Vive User Guide.} [Online]. Retrieved April 19, 2017 from \url{https://dl4.htc.com/Web_materials/Manual/Vive/Vive_User_Guide.pdf}

\harvarditem{Steam, The Body VR}{2017}{the-body-vr-pic}
\emph{The Body VR.} [Online]. Retrieved April 19, 2017 from \url{http://cdn.edgecast.steamstatic.com/steam/apps/451980/ss_e3e4e6850f6b76e1974aad3f65ea4a0eb04785d4.jpg}

\harvarditem{Steam, Job Simulator}{2017}{job-simulator-pic}
\emph{Job Simulator.} [Online]. Retrieved April 19, 2017 from \url{http://cdn.edgecast.steamstatic.com/steam/apps/448280/ss_ff7151cab14752f2b6501c31c3c79235b79cc45a.jpg}

\harvarditem{Steam, Calcflow}{2017}{calcflow-pic}
\emph{Calcflow.} [Online]. Retrieved April 19, 2017 from \url{http://cdn.edgecast.steamstatic.com/steam/apps/547280/ss_072b60fc7903197ffcb747600b90a3ccd1108e85.jpg}


%\harvarditem{Barančok et al.}{1995}{barancok}
%BARANČOK, D. et al. 1995. \emph{The effect of semiconductor surface
%treatment on LB film/Si interface.} In:~Physica Status Solidi (a), 
%ISSN 0031-8965, 1995, vol. 108, no.~2, \mbox{pp. K~87--90}
%
%\harvarditem{Benčo}{2001}{benco}
%BENČO, J. 2001. \emph{Metodológia vedeckého výskumu.} Bratislava~:
%IRIS, 2001, ISBN 80\discretionary{-}{-}{-}89018-27-0
%
%\harvarditem{Gonda}{2001}{gonda}
%GONDA, V. 2001. \emph{Ako napísať a~úspešne obhájiť diplomovú prácu.}
%Bratislava~: Elita, 2001, 3. doplnené a~prepracované vydanie, 120~s.
%ISBN 80-8044-075-1
%
%\harvarditem{Jadr. fyz. a~tech.}{1985}{slovnik}
%\emph{Jadrová fyzika a~technika: Terminologický výkladový slovník.}
%2.~rev.~vyd. Bratislava~: ALFA, 1985. 235~s. ISBN 80-8256-030-5
%
%\harvarditem{Katuščák}{1998}{kat}
%KATUŠČÁK, D. 1998. \emph{Ako písať vysokoškolské a~kvalifikačné
%práce.} Bratislava~: Stimul, 1998, 2.~doplnené vydanie. 121~s. ISBN
%80-85697-82-3
%
%\harvarditem{Lamoš a~Potocký}{1989}{lamos}
%LAMOŠ, F. -- POTOCKÝ, R. 1989. \emph{Pravdepodobnosť a~matematická
%štatistika.} 1.~vyd. Bratislava~: Alfa, 1989. 344~s. ISBN 80-8046-020-5
%
%\harvarditem{Rejtharová a~Skálová}{1981}{rejtharova}
%REJTHAROVÁ, V. -- SKÁLOVÁ, E. 1981. \emph{P\v{r}íručka anglického
%odborného stylu.} Praha~: Academia, 1981, 220~s.
%
%\harvarditem{Sýkora a~i.}{1980}{sykora}
%SÝKORA, F. a~iní. 1980. \emph{Telesná výchova a~šport.} 1.vyd.
%Bratislava~: SPN, 1980. 35~s. ISBN 80-8046-020-5
%
%\harvarditem{Steinerová}{2000}{steinerova}
%STEINEROVÁ, J. 2000. \emph{Základy filozofie človeka v~knižničnej
%a~informačnej vede.} In:~Kimlička, Š., Knižničná a~informačná veda na
%prahu informačnej spoločnosti. Bratislava~: Stimul, 2000. ISBN
%80-2274-035-2, s. 327--334
%
%\harvarditem{Šumichrast}{1995}{sumichrast}
%ŠUMICHRAST, Ľ. 1995. \emph{On the performance of higher approximations
%of radiation boundary conditions for the simulation of wave propagation
%in structures of integrated optics.} In:~Photonics '95. Prague~: CTU,
%1995, pp. 159--161
\end{thebibliography}
%
\section*{Appendices}
\addcontentsline{toc}{section}{\numberline{}Appendices}
\thispagestyle{empty}

\begin{description}
	\item[Appendix A] CD with MathworldVR code and Chromium browser
	\item[Appendix B] System manual
	\item[Appendix C] User manual
\end{description}
%
\section*{Appendix A}
\addcontentsline{toc}{section}{\numberline{}Appendix A}
%%
\subsection*{System manual}

\subsubsection*{Hardware specification}
To use HTC Vive, your computer must meet the following minimum system requirements.

\begin{itemize}
\item{GPU: NVIDIA® GeForce® GTX 970, AMD Radeon™ R9 290 equivalent or better}
\item{CPU: Intel® Core™ i5-4590/AMD FX™ 8350 equivalent or better}
\item{RAM: 4 GB or more}
\item{Video output: HDMI 1.4, DisplayPort 1.2 or newer}
\item{USB port: 1x USB 2.0 or better port}
\item{Operating system: Windows® 7 SP1, Windows® 8.1 or later, Windows® 10}
\end{itemize}

\subsubsection*{HTC Vive installation}
Detailed installation manual for HTC Vive virtual reality headset can be found at \url{https://dl4.htc.com/Web_materials/Manual/Vive/Vive_User_Guide.pdf} or by searching the Support page at \url{https://www.vive.com/us/support/}.

HTC Vive comes in a set including headset, base stations called \textsl{Lighthouses} for creating the laser field for precision tracking, hand controllers, USB and power cables.

\subsubsection*{Steam and SteamVR installation}
For HTC Vive to work, user need to install \textsf{Steam} and \textsf{SteamVR}.

System requirements:
\begin{itemize}
\item{Windows XP, Vista, 7, 8, 8.1 or 10.}
\item{512MB RAM minimum.}
\item{1GHz CPU or better.}
\item{1GB of free space on disk.}
\item{Internet connection.}
\end{itemize}

To download Steam, go to \url{http://store.steampowered.com/about/} and click on green button shown on the image below:

\begin{figure}[ht!]
\centering
\includegraphics[width=0.8\textwidth]{steam}
\caption{Steam download page.}
\label{r:71}
\end{figure}

After Steam download is finished, you can open the installer and follow instructions.

For SteamVR, open installed Steam, navigate to \textsf{Library} and search for "steamvr". It should find the installer and show it in list view under \textsf{Tools}. Clicking on it will show a button in the right panel. Click on it to start SteamVR download, installation will begin automatically afterwards.

\begin{figure}[ht!]
\centering
\includegraphics[width=0.8\textwidth]{steam-vr}
\caption{SteamVR installation button.}
\label{r:72}
\end{figure}

\newpage
\subsubsection*{Room-scale tracking setup}
After SteamVR is installed, room setup window will pop up automatically. For MathworldVR to work correctly, you need to select \textsl{Room-Scale} option and follow the tutorial.

\begin{figure}[ht!]
\centering
\includegraphics[width=0.9\textwidth]{room-scale-setup}
\caption{Room setup window.}
\label{r:73}
\end{figure}

\subsubsection*{WebVR-supported browser}
During the time of writing of this thesis, only Chromium (experimental version of Google Chrome) and Mozilla Firefox Nightly browsers supported WebVR API. We included Chromium browser with MathworldVR application on the appended CD.

Chromium doesn't need to be installed. To open it, user just needs to double-click on \texttt{chrome.exe} file. In order to enable access to the WebVR APIs in Chromium, user must select the "Enabled" option from the drop-down menu for the "Enable WebVR" flag (enter \texttt{chrome://flags/#enable-webvr} in the URL bar) and the "Gamepad Extensions" flag (enter \texttt{chrome://flags/#enable-gamepad-extensions} in the URL bar), or launch Chromium from the command line with the \texttt{--enable-webvr} and \texttt{--enable-gamepad-extensions} options.

\begin{figure}[ht!]
\centering
\includegraphics[width=0.9\textwidth]{chromium-flag}
\caption{Enabling the WebVR API in Chromium browser.}
\label{r:74}
\end{figure}



%
\section*{Appendix B}
\addcontentsline{toc}{section}{\numberline{}Appendix B}
%\appendix
%\section{Príloha}
\subsection*{User manual}
MathworldVR user manual gives an assistance to people using it. It includes described images of VR headset, controllers usage and screenshots of virtual user interface.

\subsubsection*{HTC Vive headset}
\begin{figure}[ht!]
\centering
\includegraphics[width=0.8\textwidth]{htc-vive-headset.png}
\caption{HTC Vive headset - technical detail. \cite{htc-vive-user-guide}}
\label{r:75}
\end{figure}

\begin{enumerate}
\item{Camera lens - can be used for viewing the "real world" from within the headset.}
\item{Tracking sensor - allowing Lighthouse base stations to "see" the headset.}
\item{Headset button - used for opening the SteamVR menu from within virtual reality view.}
\item{Status light - if the light produces red color, headset is not ready; if the light produces green color, headset is ready to be used.}
\item{Lens distance knob - used for changing the distance between lenses.}
\end{enumerate}

\newpage
\subsubsection*{Hand controllers}
\begin{figure}[ht!]
\centering
\includegraphics[width=0.9\textwidth]{vive-controllers.png}
\caption{HTC Vive hand controllers - technical detail. \cite{htc-vive-user-guide}}
\label{r:76}
\end{figure}

\begin{enumerate}
\item{Menu button - This button doesn't have any effect in MathworldVR.}
\item{Trackpad - Allows user to move in scrolling menus, just like double-finger gesture on notebook trackpads. It's also clickable. Clicking and holding the trackpad on the left controller will trigger teleportation beam. User can point the beam where he wants to be teleported and then release the trackpad.}
\item{System button - Used for opening the SteamVR menu.}
\item{Status light - Indicates the status of a controller. Green color means it's ready, blinking orange color means it needs to be recharged and blue color means it needs to be synced with the Lighthouse base stations, because they cannot see the controller.}
\item{Micro-USB port - Used for recharging the controller and also to upgrade its firmware.}
\item{Tracking sensor - Thanks to this sensor, Lighthouse base stations can see the controller. Tracking of HTC Vive controllers is very precise, because Lighthouse base stations emits laser beam that maps the space in front of them.}
\item{Trigger button - Used for selecting the options and moving the sliders in SettingsPanel component.}
\item{Grip button - Used for grabbing and scaling the parmetrized function. For grabbing interaction, only one of two controller's grip buttons needs to be pressed. For scaling interaction, both left and right controller's grip buttons needs to be pressed, scaling the grabbed object up and down according to movement of both controllers.}
\end{enumerate}

\newpage
\subsubsection*{Interacting with parametrized function}
Parametrized function can be grabbed, scaled and its variables changed in settings panel. To grab the function, you need to press the grip button on one of your VR controllers. You can move and rotate the function freely while grabbing it, just like objects in real life. To scale the function, you need to press grip buttons on both VR controllers imultaneously and move them fro or to each other. Moving controllers to each other will make the function smaller. Moving controllers from each other will make the function bigger.

\begin{figure}[ht!]
\centering
\includegraphics[width=0.9\textwidth]{pfunc-component.png}
\caption{ParametrizedFunction component.}
\label{r:77}
\end{figure}

\newpage
\subsubsection*{Changing function variables and color}
To change the function variables or its color, you can use the SettingsPanel component. Pointing the VR controller at this panel will initiate the laser pointer. Move the pointer at any slider and press the trigger button. Holding and moving the controller will change the value of selected variable. You can also press and hold the grip button while pointing at SettingsPanel - this will automatically move the panel on top of your controller.
 
\begin{figure}[ht!]
\centering
\includegraphics[width=0.9\textwidth]{settings-component.png}
\caption{Settings component.}
\label{r:78}
\end{figure}

\newpage
\subsubsection*{User input}
You can change right side of the equation by simply clicking the buttons on virtual calculator. This component was designed to resemble real calculator and make user interaction with it more intuitive. After you write the function, you can update the 3D visualization by clicking the \textsl{Update} button on virtual calculator.

\begin{figure}[ht!]
\centering
\includegraphics[width=0.9\textwidth]{calculator-component.png}
\caption{Calculator component.}
\label{r:78}
\end{figure}
%\include{appendixc}
%% begin the 'Curriculumvitae' of the author
%\curriculumvitae\protect\label{page:posledna}
%Táto časť\/ je nepovinná. Autor tu môže uviesť\/ svoje biografické
%údaje, údaje o~záujmoch, účasti na~projektoch, účasti na~súťažiach,
%získané ocenenia, zahraničné pobyty na~praxi, domácu prax, publikácie
%a~pod.
%\endcurriculumvitae

\end{document}
%%