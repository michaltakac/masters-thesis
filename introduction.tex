\setcounter{page}{1}
\setcounter{equation}{0}
\setcounter{figure}{0}
\setcounter{table}{0}

\section*{Introduction}
\addcontentsline{toc}{section}{\numberline{}Introduction}
Mathematics is everywhere around us. For technical science it is an important part of the study. It is necessary to understand the context, discover it gradually and build the knowledge step by step. There is also a lot of literature that often approaches mathematics in the very human language. But still, many people are afraid of it, have prejudices and do not even try to master math. 

In this thesis we're introducing novel learning environment with the~use of virtual reality on the~web that can be used as possible way of teaching mathematics. It's an attempt to make studying mathematics more interesting and fun.

Our experimental project is called MathworldVR, which sets to explore the~possibilities and introduce novel methods of using web technologies for creating room-scale, immersive learning environment in virtual reality for helping students to explore, learn about and experiment with various parametrized functions. It’s also a~practical tool for teachers to showcase abstract concepts in concrete 3D environment during lectures.

Thesis starts with the~"Problem expression" section, where we describe our goal and methodology of how we're approaching the~development of MathworldVR. In "Analysis of current state" section, current state of virtual reality is described with introduction to the~technology, examples of different VR applications, what development platform and tools are used for their development and advantages of VR on the~web, called WebVR in short.

In "Technologies and tools for development", we provide detailed information about technologies used for building the~MathworldVR application. The~selection of technologies has it's meaning - all of them are used in modern web development by big companies and/or was created by them. All of them are also open-source, because MathworldVR will be open-sourced as well.

Design and functionality planning is explained in "Software design of virtual reality application" section, describing what decisions were made when architecting and designing MathworldVR and what functionality user can expect.

In "Implementation of virtual reality application", project setup, development process, testing and deployment is explained with detailed description of individual components that together makes the~MathworldVR experience possible.

Findings and future project roadmap is discussed in "Conclusion" section.