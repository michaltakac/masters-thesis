\section{Analysis of current state of virtual and augmented reality}

\subsection{Building virtual reality applications and experiences}

The leading platform for building VR experiences today is the game engine Unity, both because the company had the foresight to add support for the Oculus Rift development kit early on, but also simply because the early use cases from when Oculus Rift was still just a very successful Kickstarter project centered around video games. 

\subsection{Virtual reality on the web}

WebVR provides support for exposing virtual reality devices — for example head-mounted displays like the HTC Vive or Oculus Rift — to web apps, enabling developers to translate position and movement information from the display into movement around a 3D scene in browser. As of today, support for both head-mounted displays is available in experimental or development builds of Chrome and Firefox, with official release planned for 2017. This has numerous very interesting applications, from virtual product tours and interactive training apps to immersive first person games.
Unity, for instance, is able to make native builds for all major platforms from the same code base, including PC, Mac, Linux, iOS, Android and more. When made by professionals, such native builds will undoubtedly look better and run faster than a comparable VR experience built with WebGL and WebVR (at least AAA games or other experiences where high fidelity and performance are paramount).

The major advantage of WebVR over natively built experiences is the same as the web has always had over desktop apps and mobile apps today - no need to download and install anything. User just needs to click a link, type in a url, and the application runs directly in her browser. There’s no app store needed. Web developers can also take advantage of many open source libraries available on the internet.