\section*{Appendix B}
\addcontentsline{toc}{section}{\numberline{}Appendix B}
%\appendix
%\section{Príloha}
\subsection*{Bibliografick\'e odkazy}

Táto časť\/ záverečnej práce je povinná. V~zozname použitej literatúry
sa uvádzajú odkazy podľa normy STN ISO 690--2 (01 0197) (Informácie
a~dokumentácia. Bibliografické citácie. Časť\/ 2: Elektronické
dokumenty alebo ich časti, dátum vydania 1.~12.~2001, ICS: 01.140.20).
Odkazy sa môžu týkať\/ knižných, časopiseckých a~iných zdrojov
informácií (zborníky z~konferencií, patentové dokumenty, normy,
odporúčania, kvalifikačné práce, osobná korešpondencia a~rukopisy,
odkazy cez sprostredkujúci zdroj, elektronické publikácie), ktoré boli
v~záverečnej práci použité.

Forma citácií sa zabezpečuje niektorou z~metód, opísaných v~norme STN
ISO 690, 1998, s.~21. Podrobnejšie informácie nájdete na stránke
\texttt{http://www.tuke.sk/anta/} v~záložke {\small\sf Výsledky
práce/Prehľad normy pre publikovanie STN ISO 690 a~STN ISO 690-2}.

Existujú dva hlavné spôsoby citovania v~texte.

\begin{itemize}
\item Citovanie podľa mena a~dátumu.
\item Citovanie podľa odkazového čísla.
\end{itemize}

\emph{Preferovanou metódou citovania} v~texte vysokoškolskej
a~kvalifikačnej práce je podľa normy ISO 7144 citovanie podľa mena
a~dátumu \citep{kat,gonda}. V~tomto prípade sa zoznam použitej
literatúry upraví tak, že za meno sa pridá rok vydania. Na uľahčenie
vyhľadávania citácií sa zoznam vytvára v~abecednom poradí autorov.

\medskip

Príklad:
\dots podľa \citep{steinerova} je táto metóda dostatočne rozpracovaná
na to, aby mohla byť\/ všeobecne používaná v~\dots

\medskip

Druhý spôsob uvedenia odkazu na použitú literatúru je uvedenie len
čísla tohto zdroja v~hranatých zátvorkách bez mena autora (autorov)
najčastejšie na konci príslušnej vety alebo odstavca.

\medskip

Príklad:
\dots podľa [13] je táto metóda dostatočne rozpracovaná na to, aby
mohla byť\/ všeobecne používaná v~\dots ako je uvedené v~[14].

\medskip

Citácie sú spojené s~bibliografickým odkazom poradovým číslom v~tvare
indexu alebo čísla v~hranatých zátvorkách. Odkazy v~zozname na konci
práce budú usporiadané podľa týchto poradových čísel. Viacero citácií
toho istého diela bude mať\/ rovnaké číslo. Odporúča sa usporiadať\/
jednotlivé položky v~poradí citovania alebo podľa abecedy.

\medskip
\noindent
Rôzne spôsoby odkazov je možné dosiahnuť\/ zmenou voľby v~balíku
\verb+natbib+:

\noindent
\verb+% Citovanie podla mena autora a roku+\\
\verb+\usepackage[]{natbib}\citestyle{chicago}+\\
\verb+% Možnosť rôznych štýlov citácií. Príklady sú uvedené+\\
\verb+% v preambule súboru natbib.sty.+\\
\verb+% Napr. štýly chicago, egs, pass, anngeo, nlinproc produkujú+\\
\verb+% odkaz v tvare (Jones, 1961; Baker, 1952). V prípade, keď+\\
\verb+% neuvedieme štýl citácie (vynecháme \citestyle{}) v "options"+\\
\verb+% balíka natbib zapíšeme voľbu "colon".+

\medskip
\noindent
Keď zapneme voľbu \verb+numbers+, prepneme sa do režimu citovania
podľa odkazového čísla.

\noindent
\verb+% Metoda ciselnych citacii+\\
\verb+\usepackage[numbers]{natbib}+

\bigskip

Pri zápise odkazov sa používajú nasledujúce pravidlá:

V~odkaze na knižnú publikáciu (pozri príklad zoznamov na konci tejto
časti):
\begin{itemize}
\item Uvádzame jedno, dve alebo tri prvé mená oddelené pomlčkou,
ostatné vynecháme a~namiesto nich napíšeme skratku et al. alebo a~i.
\item Podnázov sa môže zapísať\/ vtedy, ak to uľahčí identifikáciu
dokumentu. Od názvu sa oddeľuje dvojbodkou a~medzerou.
\item Dlhý názov sa môže skrátiť\/ v~prípade, ak sa tým nestratí
podstatná informácia. Nikdy sa neskracuje začiatok názvu. Všetky
vynechávky treba označiť\/ znamienkami vypustenia  \uv{\dots}
\end{itemize}

Pri využívaní informácií z~elektronických dokumentov  treba
dodržiavať\/ tieto zásady:
\begin{itemize}
\item  uprednostňujeme autorizované súbory solídnych služieb
a~systémov,
\item zaznamenáme dostatok informácií o~súbore tak, aby ho bolo opäť\/
možné vyhľadať\/,
\item urobíme si kópiu použitého prameňa v~elektronickej alebo
papierovej forme,
\item za verifikovateľnosť\/ informácií zodpovedá autor, ktorý sa na
ne odvoláva.
\end{itemize}

Pre zápis elektronických dokumentov platia tie isté pravidlá, ako pre
zápis \uv{klasických}. Navyše treba uviesť\/ tieto údaje:
\begin{itemize}
\item  druh nosiča  [online], [CD-ROM], [disketa], [magnetická páska]
\item dátum citovania  (len pre online dokumenty)
\item dostupnosť\/  (len pre online dokumenty)
\end{itemize}

Poradie prvkov odkazu je nasledovné:
Autor. Názov. In Názov primárneho zdroja: Podnázov. [Druh  nosiča].
Editor. Vydanie alebo verzia. Miesto vydania : Vydavateľ, dátum
vydania. [Dátum citovania]. Poznámky.  Dostupnosť\/. ISBN alebo ISSN.