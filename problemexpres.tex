%%
\section{Theoretical analysis}
Undoubtedly VR has attracted a lot of interest of people in last few years. Being a new paradigm of user interface it offers great benefits in many application areas. It provides an easy, powerful, intuitive way of human-computer interaction. The user can watch and manipulate the simulated environment in the same way we act in the real world, without any need to learn how the complicated (and often clumsy) user interface works. Therefore many applications like flight simulators, architectural walkthrough or data visualization systems were developed relatively fast. Later on, VR has was applied as a teleoperating and collaborative medium, and of course in the entertainment area.

One can say that virtual reality established itself in many disciplines of human activities, as a medium that allows easier perception of data or natural phenomena appearance. Therefore the education purposes seem to be the most natural ones. The intuitive presentation of construction rules (virtual Lego-set), visiting a virtual museum, virtual painting studio or virtual music playing \citep{loeffler} are just a few examples of possible applications. 

Virtual environments are inherently three-dimensional. They can provide interactive playgrounds with a degree of interactivity that goes far beyond what is possible in reality. If using VR as a tool for mathematics education, it ideally offers an added benefit to learning in a wide range of mathematical domains \citep{kaufmann}.
