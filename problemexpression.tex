\section{Problem expression}
The~main goal of master's thesis with the~name "Application of Virtual and Augmented Reality in Education" is to conceptualize, design and implement a virtual learning environment (VLE) for higher education, primarily focused on mathematics. This environment will be available on the~web in form of client-side, web application and accessible through desktop and mobile web browser. Main emphasis will be put into desktop platform, since currently, it's the~only platform that supports interactive VR capabilities of six degrees of freedom, head-mounted display (HMD) and hand controllers. We're specificaly focused on consumer version of HTC Vive (HMD) since we have one at our disposal and on which we'll be testing the~implementation of application.

\subsection{Methodology of the~thesiss}
Development of MathworldVR web application will include these steps:
\begin{itemize}
\item{At the~beginning it's necessarry to become familiar with current state of VR support on the~web and WebVR frameworks. This will require us to study the~available WebVR API and it's state of implementation in modern, popular browsers.}
\item{After acknowledgement of limitations and possibilities of web browsers we can proceed to conceptualization and design. We need to define the~functionality of MathworldVR, criteria for application and select the~approach to development.}
\item{Then we can proceed to selecting a~WebVR framework.}
\item{Before start of the~development, we need select the~technologies. MathworldVR will be client-side WebVR application and since WebVR is a~novel technology, we need to look into modern development tools and approaches.}
\item{After selecting the~development tools we need to prepare the~development PC for work by installing Node.js, NPM, initialize the~project with NPM, install needed development tools and project dependencies in form of Node modules. For our needs, we'll go with code editor/IDE "VSCode" by Microsoft, which includes tools for efficient programming and static code analysis.}
\item{MathworldVR will be deployed to web server and tested manually in different web browsers that support WebVR API with use of HTC Vive HMD to ensure compatibility and good user experience.}
\item{In the~end, MathworldVR is tested privately by students at Technical University in Košice and lecturers from the~department of mathematics at FBERG.}
\end{itemize}
