\section{Conclusion}
Goal of the thesis was to explore possibilities of virtual reality on the web, design and implement VR application called MathworldVR that could help students explore, experiment and learn about various parametrized functions, allowing them to build an intuition of how such functions behave according to changes of their variables.

We managed to not only successfuly implement the WebVR application, but also push forward the whole ecosystem of web-based virtual reality provided by A-Frame framework. We decided to open-source MathworldVR so more ideas can "flow" freely into the project through contributors from all around the world. Public code repository be found on this link: \url{https://github.com/michaltakac/mathworldvr}

Many components were created from scratch for the project and we also inspired other A-Frame developers by sharing the progress along the way on social media websites Facebook, Twitter and Instagram.

MathworldVR cought an eye of prof. Steven Abbott (\url{https://www.stevenabbott.co.uk/}), former Visiting Professor at the School of Mechanical Engineering at University Leeds, who helped us with the Oculus Touch support.

We also joined Virtuleap's online, worldwide WebVR hackathon (\url{http://www.virtuleap.com/}), where we managed to move into finals from 33 participanting projects and ended in 10th place overall. This triggered the public interest, helping get MathworldVR mentioned in A-Frame weekly blog and VentureBeat, \textsl{"the leading source for news, events, groundbreaking research and perspective on technology innovation"} (\url{https://venturebeat.com/2017/01/10/virtuleap-hackathon-generates-a-bunch-of-webvr-projects/}).

MathworldVR was tested by students of 2nd level from Faculty of Mining, Ecology, Process Control and Geotechnology at Technical University of Košice at the last lecture of \textsl{Modern Trends in Informatics} subject. Their response was very positive. They actively sought for more information about the project and some of them shown interest in handing out a helping hand in future.

Current functionality of MathworldVR is very limited, but interest and recognition this project received shown that it's definitely worth to take it to the next level in near future. This will be also easier since we open-sourced it. 

Next step will be building the infrastructure, adding the multi-user support and creating a back-end with database. We're looking into Elixir programming language and Phoenix framework, because it looks very promising. It leverages the ability of Erlang's virtual machine to handle millions of connections alongside Elixir's beautiful syntax and productive tooling for building fault-tolerant systems. This will make creating different (public or private) "rooms" and persisting the position, rotation and application state, possible.

Another planned features include: more examples in form of "mathematical rooms" of different kind, better calculator component, intuitive menu added to VR hand controller incorporating multiple options and layers, 2D graphs support, double and surface integrals, vectors and many more. We plan to create a public roadmap of planned features to serve as a guide for developers, universities and others who are interested about the progress.